\documentclass[11pt,a4paper]{article}
\usepackage[left=2cm,right=2cm,top=2cm,bottom=2.3cm]{geometry}
\usepackage{amssymb,amsmath}
\usepackage[unicode,colorlinks]{hyperref}
\usepackage[pdftex]{graphicx}
\usepackage{wrapfig}
\usepackage{subfig}
\usepackage{calc}
\usepackage{tikz}
\usepackage{settobox}
\usepackage{pgfplots}
\usetikzlibrary{arrows}
\usetikzlibrary{calc}
\usetikzlibrary{%
    decorations.pathreplacing,%
    decorations.pathmorphing,%
    decorations.markings%
}


\usepackage[utf8]{input enc}
\usepackage[english, russian]{babel}

\numberwithin{equation}{section}

\newcommand{\nn}{\nonumber}
\newcommand{\pt}{\partial}
\newcommand{\grad}{\mathrm{grad}\,}
\newcommand{\rot}{\mathrm{rot}\,}
\renewcommand{\div}{\mathrm{div}\,}
\newcommand{\vn}{\vec{\nabla}}
\newcommand{\eps}{\epsilon}
\newcommand{\vareps}{\varepsilon}
\newcommand{\const}{\mathrm{const}}
\newcommand{\com}[1]{{\Large{\texttt{{\color{red}(#1)}}}}}

\tikzset{>=latex}
\graphicspath{{./pics/}}
\title{Электродинамика}
\author{Игорь Шендерович\\\texttt{shender.i@gmail.com}}
\begin{document}
\maketitle

\section{Векторный анализ.}

Рассмотрим электрический заряд $q$, расположенный в какой-нибудь точке
в пространстве. Этот заряд создаёт статическое электрическое поле. По
закону Кулона его напряжённость выражается формулой
\begin{equation}
  \label{eq:coulomb}
  \vec{E}(\vec{r}) =k \frac{q \vec{r}}{r^3}.
\end{equation}

Видно, что вектор $\vec{E}$ существует в любой точке пространства, вне
зависимости от того, насколько далеко мы отошли от заряда. Это
позволяет ввести понятие \textit{векторного поля} --- вида материи,
который существует при наличии источника. В данном случае источником
электрического поля является заряд $q$. Позднее мы увидим, что
магнитное поле, несмотря на отсутсвие одиночных источников, также
допускает такую интерпретацию. С этого момента мы будем говорить не о
напряжённости $\vec{E}$, а об электрическом (или магнитном) поле
$\vec{E}(x,y,z,t)$ (или $\vec{B}(x,y,z,t)$). Заметим, что поле может
зависеть как от точки пространства, так и от времени. 

Поскольку вектор определяется своими проекциями, то задать векторное
поле --- то же самое, что задать три его проекции. Таким образом,
электрическое поле $\vec{E}$ --- три функции четырёх переменных. 

Понятно, что в природе существуют и другие поля, не обязательно
векторные. Рассмотрим, например, поле температур $T(x,y,z,t)$. Это
тоже поле (т.к. температуру можно определить в любой точке
пространства), но \textit{скалярное}, поскольку температура не имеет
направления, а определяется лишь числовым значением. 

Другой пример векторного поля, более приближенный к реальности ---
поле скоростей жидкости. Каждой <<частице>> движущейся жидкости можно
сопоставить векторную функцию $\vec{v}(\vec{r},t)$, которая описывает
скорость данной частички. По определению, это векторное поле
скоростей. Мы увидим в дальнейшем, что многие свойства этого поля
жидкости переносятся и на электродинамику. 

Электрические и магнитные (или просто электромагнитные) поля устроены
довольно сложно, но при этом связь между значениями полей в двух
соседних точках довольно проста. Задача наших упражнений --- вывести
эту связь в наиболее общем виде. 

\begin{wrapfigure}{r}{40mm}
  \vspace{-1cm}
  \begin{center}
  \includegraphics[width=4cm]{lines.pdf}  
  \end{center}
  \vspace{-1cm}
  \caption{Силовые линии.}
  \label{fig:force_lines}
  \vspace{-1.1cm}
\end{wrapfigure}

Как можно зрительно представлять поля? Лучше всего это делать с
помощью \textbf{силовых линий} --- таких линий, касательные к которым в
каждой точке будут давать направление вектора напряжённости в этой
точке. 

Чтобы изобразить на подобной картинке величину модуля вектора
напряжённости, можно условиться рисовать линии гуще в тех местах, где
абсолютная величина этого вектора больше. 

\subsection{Поток.}
\label{sec:flux}

Векторные поля обладают двумя очень важными характеристиками, которые
мы будем использовать при описании. Первая из них --- \textbf{поток}. 

Рассмотрим, к примеру, поток жидкости через некоторую ограниченную
поверхность. Можно задать себе вопрос --- сколько жидкости втекает
(вытекает) через эту поверхность площади $S$? 

Это количество можно посчитать следующим образом. Разобъём нашу
поверхность на много кусочков с площадью $dS$ каждый. К каждому
кусочку проведём вектор нормали $\vec{n}$ (так, чтобы он смотрел
наружу поверхности, а не внутрь). Тем самым, у каждого кусочка площади
будет задана ориентация. Для краткости совокупность данных о площади
кусочка и векторе нормали можно записывать в виде одного вектора
$d\vec{S}$ --- это вектор, по модулю равный $dS$, с направлением,
совпадающим с $\vec{n}$.

\begin{wrapfigure}{r}{40mm}
  \vspace{-1cm}
  \begin{center}
    \includegraphics[width=4cm,height=4cm]{flux.pdf}
  \end{center}
  \vspace{-0.7cm}
  \caption{Поток.}
  \label{fig:flux}
\end{wrapfigure}

Спроецируем на направление этого вектора наше поле $\vec{v}$. Операция
проектирования проще всего выглядит как скалярное произведение
$\vec{v} \cdot d \vec{S}$. Действительно, по определению, скалярное
произведение равно $d\Phi = v \cdot dS \cdot \cos \alpha$, где $\alpha$ ---
угол между вектором $\vec{v}$ и нормалью $\vec{n}$. Теперь
просуммируем подобные выражения по всей поверхности, иными словами,
проинтегрируем по ней:

\begin{equation}
  \label{eq:def_flux}
  \Phi \equiv \int_S \vec{v} \cdot d \vec{S}.
\end{equation}

По определению, это скалярное произведение и называется потоком $\Phi$ 
векторного поля $\vec{v}$. Можно интерпретировать это определение и
таким способом: поток векторного поля равен среднему значению
нормальной компоненты поля $\vec{v}$, умноженному на величину площади
$S$. 

Если, например, поле устроено так, что не зависит от конкретной точки
(всюду постоянное поле), то формула для расчёта потока может быть
сильно упрощена. Пусть, скажем, поле всюду направлено по нормали к
поверхности (пример --- электрическое поле точечного заряда, а
поверхность --- сфера). Тогда в любой точке скалярное произведение
превращается в обычное (т.к. $\cos 0 = 1$), и функцию $v$ можно
вынести за знак интеграла. Интеграл же от $dS$ даст просто площадь
поверхности. Таким образом, в этом простейшем случае поток станет
равен $\Phi = vS$. 

В более сложных случаях (которые обычно и встречаются в природе),
вычисление потока по определению --- довольно сложная задача. К
счастью, у нас будет \textbf{теорема Гаусса}, которая позволяет свести
вычисление потока к относительно простым операциям. 

\subsection{Циркуляция.}
\label{sec:curl}

Опять представим себе поле скоростей, описывающее поток
жидкости. Зададимся таким вопросом: циркулирует ли эта жидкость? Иными
словами, существует ли её вращательное движение вдоль некоторого
замкнутого геометрического контура? 

\begin{wrapfigure}{r}{40mm}
  \vspace{-1.5cm}
  \begin{center}
    \includegraphics[width=4cm,height=4cm]{curl.pdf}
  \end{center}
  \vspace{-0.7cm}
  \caption{Циркуляция.}
  \label{fig:curl}
\end{wrapfigure}


По определению, \textbf{циркуляцией} $C$ называется скорость жидкости вдоль
контура, умноженная на длину этого контура. Как и в случае с потоком,
можно думать про эти величины как про средние; более аккуратное
определение можно получить, опять пользуясь понятием проекции. 

Рассмотрим участок контура длиной $dl$. С ним, как и с кусочком
площадки, можно связать направление. Рассмотрим вектор $d\vec{l}$ ---
это вектор, модуль которого равен $dl$, а направление совпадает с
направлением касательной в данной точке. Теперь спроецируем наше
векторное поле $\vec{v}$ на этот вектор $d\vec{l}$. Сделать это можно
так же, как и в случае с потоком, т.е. взять скалярное
произведение. Теперь просуммируем это по всему контуру:

\begin{equation}
  \label{eq:curl}
  C \equiv \oint_\Gamma \vec{v} \cdot d \vec{l}.
\end{equation}

Пользуясь понятиями потока и циркуляции, мы опишем все законы
электродинамики, т.е. получим уравнения Максвелла. 

\subsection{Произведение векторов.}
\label{sec:vector_product}

Разберём подробнее операции с векторами. Нам понадобится умение
перемножать вектора двумя способами, дифференцировать их и
интегрировать. 

Два вектора $\vec{a}$ и $\vec{b}$ можно перемножить двумя способами:
скалярно и векторно. \textbf{Скалярное произведение} определяется через
компоненты этих векторов:

\begin{equation}
  \label{eq:def_scalar_product}
  \vec{a} \cdot \vec{b} \equiv a_x b_x + a_y b_y + a_z b_z.
\end{equation}

С помощью скалярного произведения можно также определить модуль
вектора: это корень из скалярного произведения вектора на самого себя.

\begin{equation}
  \label{eq:def_modulus_vector}
  |a| \equiv \sqrt{\vec{a} \cdot \vec{a}}.
\end{equation}

Также можно образовать \textbf{векторное произведение}: такое
произведение двух векторов, результатом которого является снова
вектор. Модуль этого вектора равен

\begin{equation}
  \label{eq:def_cross_product}
  | \vec{a} \times \vec{b}| = |\vec{a}| |\vec{b}| \sin \alpha,  
\end{equation}
а направление задаётся правилом правой руки. Угол $\alpha$ --- угол
между векторами. 

\begin{wrapfigure}{r}{40mm}
  \vspace{-1.5cm}
  \begin{center}
    \includegraphics[width=4cm,height=4cm]{cross.pdf}
  \end{center}
  \vspace{-0.7cm}
  \caption{Векторное произведение.}
  \label{fig:curl}
  \vspace{-1cm}
\end{wrapfigure}

Свойства этого произведения довольно просты. Во-первых, если вектора
$\vec{a}$ и $\vec{b}$ коллинеарны, то $\vec{a} \times
\vec{b}=0$. Во-вторых, модуль этого вектора совпадает с площадью
параллелограмма, натянутого на вектора $\vec{a}$ и $\vec{b}$.

А можно ли записать определение векторного произведения в координатах,
подобно \eqref{eq:def_scalar_product}? Оказывается, можно. Именно, 

\begin{eqnarray}
  \label{eq:def_cross_product_components}
  \nn
  (\vec{a} \times \vec{b})_x &=& a_y b_z -a_z b_y,\\
  (\vec{a} \times \vec{b})_y &=& a_z b_x -a_x b_z,\\
  \nn
  (\vec{a} \times \vec{b})_z &=& a_x b_y -a_y b_x.
\end{eqnarray}

Запомнить это правило довольно просто: для i-ой компоненты нужно
устроить циклическую перестановку из индексов (xyz), где нужный индекс
стоит на i-ом месте. 

Понять, откуда взялось это правило, довольно просто. Рассмотрим три
орта $\vec{i},\vec{j},\vec{k}$. Применяя к ним правило
\eqref{eq:def_cross_product}, получим, что $\vec{i} \times \vec{j} =
\vec{k}$, \ldots. Расписывая вектор $\vec{a} = a_x \vec{i} + a_y
\vec{j} + a_z \vec{k}$, а вектор $\vec{b}$ аналогично, получим
требуемые свойства \eqref{eq:def_cross_product_components}. 

\subsection{Градиент. Оператор $\nabla$.}
\label{sec:gradient}

Предположим, что у нас имеется для начала скалярное поле, типа поля
температур. Мы хотим как-то описать изменение температуры $T(x,y,z,t)$
в пространстве. В отличие от похожей задачи --- изменения температуры
по времени --- нам нужно дифференцировать не по времени, а придумать
что-то другое, более подходящее к данному случаю.

Понятно, что если бы у нас была одна координата, то сработала бы
производная $dT/dx$, потому что именно она бы определяла скорость
изменения температуры вдоль оси $x$. В данном случае нам понадобятся
три производных 

\begin{equation}
  \label{eq:def_grad_1}
  \frac{\pt T}{\pt x}, \quad   \frac{\pt T}{\pt y}, \quad   \frac{\pt
    T}{\pt z},
\end{equation}
из которых можно сделать вектор. Вектор можно соорудить естественным
способом. Вспомним, что у нас есть три орта
$\vec{i},\vec{j},\vec{k}$. Построим вектор \textbf{градиента} по такому
правилу:

\begin{equation}
  \label{eq:def_grad_2}
  \grad T (x,y,z) \equiv \nabla T \equiv \frac{\pt T}{\pt x} \vec{i} +  \frac{\pt T}{\pt y}
  \vec{j} +  \frac{\pt T}{\pt z} \vec{k}.
\end{equation}

Таким образом, можно сказать, что градиент скалярного поля --- это
аналог производной, только в числе измерений больше одного. 

Заметим, что градиент, будучи вектором, явно зависит от
направления. Это сооветствует тому факту, что температура в разных
направлениях пространства может вестии себя по-разному. Таким образом,
градиент температуры $\grad T$ --- векторное поле, образованное из скалярного
поля самой температуры $T$.

Физический смысл градиента такой: в каждой точке пространства он
указывает направление, в котором температура меняется быстрее всего. 

Оказывается, что у градиента есть ещё одна интерпретация. Она ведёт к
обобщению известной теоремы \textbf{Ньютона--Лейбница}. Эта теорема
устроена примерно так. Представим себе, что у нас есть материальная
точка, двигающаяся в пространстве. Её координаты мы будем
характеризовать радиус--вектором $\vec{r}(t)$. Вычислим скорость этой
материальной точки; по определению, она равна $\vec{v} (t) = d
\vec{r}(t) / dt$. Что получится, если мы теперь проинтегрируем эту
скорость по времени от момента $t_1$ до $t_2$? 

\begin{equation}
  \label{eq:newton_leibnitz}
  \int_{t_1}^{t_2} \vec{v}(t) \, dt = \int_{t_1}^{t_2}
  \frac{d\vec{r}(t)} {dt} \, dt = \int_{t_1}^{t_2} d\vec{r}(t) =
  \vec{r}(t_2) - \vec{r}(t_1) = \Delta \vec{r}.
\end{equation}

То есть, если проинтегрировать скорость, получится перемещение. Более
общо: если проинтегрировать производную некоторой функции $f(t)$,
получится изменение этой самой функции $f(t)$. 

Эта теорема естественным образом ограничена на прямую линию (интеграл
вдоль прямой). Что будет, если мы захотим проинтегрировать что-нибудь
(а точнее, какой--нибудь вектор) вдоль произвольной кривой? 

\begin{wrapfigure}{r}{40mm}
  \vspace{-1.2cm}
  \begin{center}
    \includegraphics[width=4cm,height=4cm]{curve_int.pdf}
  \end{center}
  \vspace{-0.7cm}
  \caption{Интеграл вдоль кривой.}
  \label{fig:curve_int}
\end{wrapfigure}


В этом случае нам нужно описать процедуру интегрирования вдоль
кривой. Рассмотрим кусочек дуги кривой $\Delta l_i$. Пускай значение функции
на нашем кусочке равно $f_i$. Тогда интегралом вдоль кривой $C$
называется такое выражение:

\begin{equation}
  \label{eq:def_curve_int_1}
  \int_C  \vec{f} \cdot d\vec{l} \equiv \lim_{N\to \infty} \sum_{i=0}^N f_i \Delta l_i. 
\end{equation}

В общем, это что-то, аналогичное стандартному определению интеграла,
только нужно учитывать, что $f_i$ --- значение функции на этом
отрезке. В нашем случае вместо скаляра $f_i$ уместно взять скалярное
произведение 

\begin{equation}
  \label{eq:def_curve_int_2}
  \grad \phi \cdot d\vec{l}_i.
\end{equation}

Мы уже знаем, что градиент показывает, насколько быстро меняется
функция в данном направлении. Таким образом, если мы спроецируем
градиент на какое-то выделенное направление, мы получим скорость
изменения функции в этом направлении. Проецирование делатся в точности
скалярным произведением \eqref{eq:def_curve_int_2}. Таким образом,
уравнение \eqref{eq:def_curve_int_2} даёт изменение скалярного поля в
направлении $d\vec{l}_i$: $\phi(i+1) - \phi(i)$, где $\phi(i)$ ---
значение поля в точке $i$. Суммируя все такие изменения, получаем, что
сумма \eqref{eq:def_curve_int_1} превращается в конечное изменение 

\begin{equation}
  \label{eq:def_curve_int_3}
  \int_C \grad \phi \cdot d\vec{l} = \phi(2) - \phi(1).
\end{equation}

Таким образом, мы видим, что у градиента есть такой смысл: будучи
проинтегрирован по какой-либо кривой, он даёт изменение конечное
изменение скалярной функции вдоль этой кривой. 

Заметим попутно, что интеграл от градиента, очевидно, не зависит от
кривой, вдоль которой мы интегрируем. Действительно, правая часть
уравнения \eqref{eq:def_curve_int_3} зависит только от значений поля
$\phi$ в точках 1 и 2, и больше не от чего. Очевидно, что значения
поля в этих точках не зависят от того, по какому пути мы в эти точки
пришли. 

Что будет, если мы придём из точки 1 в точку 2 по красному пути, а
уйдём обратно по синему? Мы опишем замкнутый контур. 

\begin{wrapfigure}{r}{40mm}
  \vspace{-1.5cm}
  \begin{center}
    \includegraphics[width=4cm,height=4cm]{path_indep.pdf}
  \end{center}
  \vspace{-0.7cm}
  \caption{Замкнутый контур.}
  \label{fig:path_indep}
\end{wrapfigure}

Пройдём по красному контуру от точки 1 до точки 2. При этом значение
интеграла \eqref{eq:def_curve_int_3} даст нам, как и положено, $\phi(2)
- \phi(1)$. Теперь пройдём от точки 2 до точки 1 по синему
контуру. Интеграл даст на этот раз $\phi(1) - \phi(2)$. В итоге наших
прогулок по разноцветным контурам мы придём обратно в точку 1, при
этом значение интеграла по замкнутому контуру будет равно 0. Опять же,
заметим, что он конкретной формы этого контура ничего не зависит. 

Таким образом, мы доказали, что интеграл от градиента по замкнутому
контуру равен нулю, и от контура не зависит. Или, вспоминая
определение циркуляции \eqref{eq:curl}, мы доказали, что циркуляция
градиента равна нулю.

\begin{equation}
  \label{eq:int_grad}
  \oint \grad \phi \cdot  d\vec{l} =0.
\end{equation}

В дальнейшем этот факт нам поможет. 

Напоследок продемонстрируем один трюк, ради которого (отчасти) это всё
и затевалось. Обозначим \emph{формально} буквой $\nabla$ такую
операцию:

\begin{equation}
  \label{eq:def_nabla}
  \vec{\nabla} \equiv \frac{\pt}{\pt x} \vec{i} +  \frac{\pt}{\pt y}
  \vec{j} +  \frac{\pt}{\pt z} \vec{k}.
\end{equation}

Буква $\vec{\nabla}$ теперь играет роль \textbf{оператора градиента},
т.е. не самостоятельной буквы, а чего-то, что имеет смысл только в
паре с функцией, к которой она применяется. Оператор градиента
<<ждёт>> функцию, которую ему надо продифференцировать. Из формулы
\eqref{eq:def_grad_2} видно, что градиент можно получить, действуя
оператором градиента на наше скалярное поле.

Очевидно, что, будучи вектором, оператор $\vec{\nabla}$ может быть применён
не только к скалярам (типа поля температуры), но и к векторам,
порождая объекты с очень прозрачным физическим смыслом. 

\subsection{Дивергенция.}
\label{sec:divergence}

Первое, что мы можем сделать с вектором $\vec{\nabla}$ и каким-то
векторным полем $\vec{A}$ --- скалярно их перемножить. Как следует из
определения, должен получиться скаляр. Посмотрим, так ли это.

\begin{equation}
  \label{eq:def_divergence}
  \div \vec{A} \equiv \vec{\nabla} \cdot \vec{A} = \frac{\pt}{\pt x}
  A_x +  \frac{\pt}{\pt y} A_y +  \frac{\pt}{\pt z} A_z = \frac{\pt
    A_x}{\pt x} +  \frac{\pt A_y}{\pt y} +  \frac{\pt A_z}{\pt z}.
\end{equation}

Скалярная величина, которую мы получили, называется
\textbf{дивергенцией}. Фактически, оператор $\vec{\nabla}$, скалярно
умноженный на $\vec{A}$, дифференцирует каждую из компонент вектора
$\vec{A}$ по соответствующей координате и складывает результаты без
учёта направления. Этим он и отличается от градиента: градиент,
напомним, даёт вектор. 

Итак, если градиент скалярному полю сопоставляет векторное, то
дивергенция векторному полю сопоставляет скалярное. 

Нет ли случайно у дивергенции какого-нибудь внятного физического
смысла? Разумеется, есть, иначе зачем бы она нам понадобилась!

Чтобы понять этот физический смысл, нам понадобится вспомнить понятие
потока \eqref{eq:def_flux}. Рассмотрим какое--нибудь тело, через
которое проходит векторное поле. Разобъём тело на маленькие одинаковые
кубики. Рассмотрим маленький кубик со сторонами $\Delta x$, $\Delta
y$, $\Delta z$. Пускай в пространстве есть какое-то векторное поле
$\vec{A}$. Вычислим его поток (наружу) через поверхность этого кубика.

У поверхности кубика 6 граней. Можно вычислить поток через каждую из
них. Поскольку грани --- маленькие квадраты, вычисление будет
несложным. Например, поток через грань 1 равен

\begin{equation}
  \label{eq:theorem_divergence_1}
  \Phi_1 = -\int A_y (x,y,z) dx dz.
\end{equation}
(знак минус появился от того, что в грань 1 наше поле \emph{входит},
так что поток \emph{наружу} будет со знаком минус). Это выражение
можно упростить: так как мы считаем наш кубик маленьким, то можно
считать, что на грани функция $A_y$ примерно постоянная; раз так, то
её можно вытащить за знак интеграла. Останется лишь интеграл по
площади грани, который равен собственно площади $\Delta x \Delta z$. 

\begin{wrapfigure}{r}{40mm}
  \vspace{-1.2cm}
  \begin{center}
    \includegraphics[width=4cm,height=4cm]{div.pdf}
  \end{center}
  \vspace{-0.7cm}
  \caption{Вычисление потока.}
  \label{fig:div}
  \vspace{2cm}
\end{wrapfigure}

Итого, получаем, что поток через грань 1 равен

\begin{equation}
  \Phi_1 = - A_y \Delta x \Delta z. 
\end{equation}

Рассмотрим теперь поток наружу через грань 3:

\begin{equation}
  \Phi_3 = \int A_y (x,y + \Delta y,z) dx dz = A_y (x,y+\Delta y,z)
  \Delta x \Delta z.
\end{equation}

Сложим теперь потоки $\Phi_1 + \Phi_3$ --- это логично сделать, потому
что у них есть много общего. 

\begin{equation}
  \label{eq:theorem_divergence_2}
  \Phi_1 + \Phi_3 = \Delta x \Delta z \left(A_y(y+\Delta y) - A_y
    (y)\right) \approx \Delta x \Delta z \Delta y \frac{\pt A_y}{\pt y}.
\end{equation}

Здесь мы заменили разность полей на двух гранях производной потому,
что кубик считается маленьким и такая замена не внесёт существенной
ошибки. Аналогичные операции можно провернуть и для остальных четырёх
граней; ответ для полного потока $\Phi$ совсем не удивителен:

\begin{equation}
  \label{eq:theorem_divergence_3}
  \Phi = \Delta x \Delta y \Delta z
  \left(
    \frac{\pt A_x}{\pt x} + 
    \frac{\pt A_y}{\pt y} + 
    \frac{\pt A_z}{\pt z}
  \right) = \Delta V \div \vec{A}. 
\end{equation}

Таким образом, физический смысл дивергенции такой: это поток
векторного поля, отнесённый к единице объёма $\Delta V$. Можно
сказать, что дивергенция меряет <<удельную величину>> потока, измеряя,
насколько мощный источник поля мы имеем. 

Более того, если мы теперь проинтегрируем равенство
\eqref{eq:theorem_divergence_3} по всему объёму тела и вспомним
определение потока \eqref{eq:def_flux}, мы получим ещё
одно соотношение, известное как \textbf{теорема Гаусса--Остроградского}:



\begin{equation}
  \label{eq:theorem_gauss_ostograd}
  \int_V \div \vec{A}\, dV =  \int_S \vec{A} \cdot  d\vec{S}.
\end{equation}

Заметим, что это соотношение чем-то похоже на теорему о градиенте
\eqref{eq:def_curve_int_3}: в правой части стоит что-то, размерности
на единицу меньшей, чем в левой части. Теперь если вы хотите сосчитать
поток какого--нибудь поля, у вас есть два варианта: либо вы честно его
считаете и берёте поверхностный интеграл, либо вы считаете дивергенцию
и берёте интеграл по объёму. Точно то же самое было в случае
градиента: либо считать интеграл вдоль кривой, либо считать приращение
функции вдоль этой же кривой.

Вскоре мы выясним, что есть ещё одна теорема похожего типа. Но для
начала посмотрим, на что ещё может сгодиться оператор $\vn$. 

\subsection{Ротор.}
\label{sec:curl}

Что ещё можно смастерить из оператора $\vec{\nabla}$ и вектора?
Помимо скалярного произведения, которое приводит к дивергенции, можно
сделать векторное произведение; в итоге получится вектор. Какие же у
него будут компоненты? Из формул \eqref{eq:def_cross_product} и
\eqref{eq:def_nabla} получаем: 

\begin{eqnarray}
  \label{eq:def_curl}
  \nn
  (\vec{\nabla} \times \vec{A})_x &=& \frac{\pt A_z}{\pt y} -  \frac{\pt
    A_y}{\pt z},\\
  (\vec{\nabla} \times \vec{A})_y &=& \frac{\pt A_x}{\pt z} -  \frac{\pt
    A_z}{\pt x},\\
  \nn
  (\vec{\nabla} \times \vec{A})_z &=& \frac{\pt A_y}{\pt x} -  \frac{\pt
    A_x}{\pt y}.
\end{eqnarray}

Вектор $\rot \vec{A} \equiv \vn \times \vec{A}$ называют
\textbf{ротором} или \textbf{вихрем}. Название выбрано неслучайно:
физический смысл этого вектора тесно связан с вихрями в потоке
жидкости (или в любом другом векторном поле).

Можно ли как-то выяснить физический смысл ротора аналогично
дивергенции и градиенту? Сейчас мы получим нечто, аналогичное теореме
Гаусса--Остроградского. Если в случае с дивергенцией мы интегрировали
поле по поверхности маленького кубика, то в случае с ротором надо
интегрировать по границе маленького квадрата (так подсказывает наша
интуиция).

\begin{wrapfigure}{r}{40mm}
  \vspace{-0.8cm}
  \begin{center}
    \includegraphics[width=4cm,height=4cm]{stokes.pdf}
  \end{center}
  \vspace{-0.7cm}
  \caption{Вычисление циркуляции.}
  \label{fig:stokes}
  \vspace{2cm}
\end{wrapfigure}


Рассмотрим, например, квадрат, расположенный в плоскость $xy$ со
сторонами $\Delta x$, $\Delta y$. Вычислим циркуляцию поля $\vec{A}$
по этому квадрату. Также как и в случае с потоком, полная циркуляция
складывается из циркуляций по каждому ребру. Посмотрим, например, на
ребро 1. Циркуляция по нему равна

\begin{equation}
  \label{eq:theorem_curl_1}
  C_1 = \int A_x (x,y) d x = A_x (x,y) \Delta x. 
\end{equation}

Здесь мы, как и в предыдущем пункте, воспользовались тем, что
квадратик маленький, а значит, поле на его ребре можно считать
постоянным вдоль ребра. Коль так, то поле можно вынести за знак
интеграла, ну а интеграл от $dx$ даст просто длину ребра. Проводя
аналогичную операцию для остальных рёбер, получим в сумме для
циркуляции

\begin{equation}
  \label{eq:theorem_curl_2}
  C_{xy} = C_1 + \ldots C_4 = \Delta x
  \left(
    A_x (x,y) - A_x(x,y+\Delta y)
  \right) + \Delta y
  \left(
    A_y (x+\Delta x, y) - A_y (x,y)
  \right).
\end{equation}

Раскладывая разность в скобках (наш контур считается малым), получим:

\begin{equation}
  \label{eq:theorem_curl_3}
  C_{xy} = \Delta x \Delta y \left(\frac{\pt A_y}{\pt x} - \frac{\pt A_x}{\pt y}\right).
\end{equation}

У нас получилось, что циркуляция поля вдоль контура, лежащего в
плоскости $xy$ равна площади этого контура $\Delta x \Delta y$,
умноженной на компоненту $z$ ротора поля $C$ (см. формулу
\eqref{eq:def_curl}). Компонента $z$ в данном случае --- нормальная
составляющая ротора по отношению к плоскости $xy$. Таким образом,
можно написать, что
\begin{equation}
  \label{eq:theorem_curl_4}
  \oint \vec{A} \cdot d\vec{l} = \rot \vec{A} \cdot \vec{n}\, \Delta S.
\end{equation}

Если теперь просуммировать циркуляции по всем маленьким контурам на
нашей поверхности $S$, получится \textbf{теорема Стокса}:

\begin{equation}
  \label{eq:theorem_curl_5_stokes}
  \oint \vec{A} \cdot d \vec{l} = \int_S \rot \vec{A} \cdot \vec{n} \,
  dS = \int_S \rot \vec{A} \cdot d\vec{S}.
\end{equation}

Опять, по аналогии с дивергенцией, заметим, что теорема связывает
нечто одномерное по сути (циркуляцию) с чем-то двумерным (интеграл по
поверхности от ротора). В каком-то смысле, теорема о градиенте
\eqref{eq:def_curve_int_3}, теорема Гаусса--Остроградского
\eqref{eq:theorem_gauss_ostograd} и теорема Стокса
\eqref{eq:theorem_curl_5_stokes} --- частные случаи одного более
общего соотношения, про которое здесь мы говорить не будем. 

\subsection{Манипуляции с $\vn$.}
\label{sec:nabla}

Сейчас мы начнём получать дивиденды от записи векторных операций через
оператор $\vn$. Попробуем составить всякие комбинации с двумя
операторами $\vn$. Например, возьмём скалярное поле $\phi$, возьмём
его градиент (получится векторное поле), от получившегося сосчитаем
ротор. Кратко это записывается так:

\begin{equation}
  \vn \times (\vn \phi).
\end{equation}

Мы можем немедленно вычислить это произведение. Дело в том, что скобки
можно расставлять как угодно (произведение ассоциативно), поэтому
получаем, что $\vn \times (\vn \phi) = (\vn \times \vn) \phi
=0$. Здесь мы воспользовались тем, что векторное произведение вектора
самого на себя равно нулю (т.к. <<угол>> между векторами равен
нулю). Итак, мы получили следующее равенство: 

\begin{equation}
  \label{eq:rot_grad}
  \vn \times (\vn \phi) = \rot \grad \phi = 0.
\end{equation}

Если мы применим к вектору $\grad \phi$ теорему Стокса
\eqref{eq:theorem_curl_5_stokes}, то получится

\begin{equation}
  \label{eq:stokes_gradient}
  \oint \grad \phi \cdot d\vec{l} = \int \rot \grad \phi \cdot
  d\vec{S} = 0.
\end{equation}

То есть, циркуляция градиента всегда равна нулю. Вспомним, что мы уже
видели это свойство в формуле \eqref{eq:int_grad}. Там мы её
доказывали напрямую, руками. А с помощью формальной операции $\vn$ и
теоремы Стокса мы получили это соотношение бесплатно.

Рассмотрим теперь комбинацию $ \vec{A} \cdot (\vec{A} \times
\vec{B})$. Вектор $\vec{A} \times \vec{B}$ перпендикулярен $\vec{A}$
(см. рис. \ref{fig:curl}), поэтому это произведение равно
нулю. Возьмём теперь в качестве $\vec{A} = \vn$, а в качестве
$\vec{B}$ какое--нибудь векторное поле $\vec{X}$. Получаем

\begin{equation}
  \label{eq:div_rot}
  \vn \cdot (\vn \times \vec{X}) = 0 = \div \rot \vec{X}.
\end{equation}

Применим к вектору $\rot \vec{X}$ теорему Гаусса--Остроградского
\eqref{eq:theorem_gauss_ostograd}:

\begin{equation}
  \label{eq:gauss_rot}
  \int_S \rot \vec{X} \cdot d\vec{S} = \int_V \div \rot \vec{X}\, dV = 0.
\end{equation}

Итак, мы получили, что поток вектора $\rot \vec{X}$ всегда равен
нулю. Это значит, что вихревое поле не создают никакие источники:
силовые линии не имеют начала и конца. Таково, например, магнитное
поле, но это мы увидим позже.

Заметим, кстати, что всё работает и в обратную сторону. Допустим, у
нас есть какое-то поле $\vec{X}$, про которое известно, что его
циркуляция по любому замкнутому контуру равна нулю (т.е. $\rot \vec{X}
=0$). Тогда всегда можно отыскать скалярную функцию $\chi$, градиент
которой даёт $\grad \chi = \vec{X}$. По теореме Стокса циркуляция
этого градиента будет равна нулю. 

Аналогично и для дивергенции: допустим, что есть какое-то поле $\vec{B}$,
дивергенция которого равна нулю. Тогда его всегда можно представить в
виде ротора другого поля $\vec{B} =\rot \vec{A}$. Эти аргументы и работают
в электродинамике. 

% В заключение отметим, что эти соотношения верны для любых полей $\phi$ и
% $\vec{V}$. Однако, не все соотношения с двумя операторами $\vn$ равны
% нулю. К примеру, мы могли подействовать на градиент не ротором, а
% дивергенцией:

% \begin{equation}
%   \label{eq:def_laplacian}
%   \vn \cdot (\vn \phi) = \vn^2 \phi = \left( \frac{\pt^2}{\pt x^2} +
%     \frac{\pt^2}{\pt y^2} + \frac{\pt^2}{\pt z^2} \right) \phi \equiv \Delta \phi.
% \end{equation}

% Оператор $\Delta$, встретившийся тут, называется \textbf{лапласианом}. В
% данном случае он переводит скалярное поле $\phi$ опять же в скаляр
% (поскольку дивергенция --- скалярное поле). 

% Раз оператор лапласиана скалярный, значит, он может действовать и на
% вектор, т.е. на каждую компоненту вектора:

% \begin{equation}
%   \vn^2 \vec{A} = \vn^2 A_x \vec{i} + \vn^2 A_y \vec{j} + \vn^2 A_z \vec{k}.
% \end{equation} 

\section{Электростатика.}
\label{sec:statics}

\subsection{Теорема Гаусса.}
\label{sec:coulomb}

После подготовительных математических процедур мы можем приступить
наконец к теории электромагнитных полей. Для начала мы будем разбирать
не общий случай, а случай, при котором все изучаемые величины не
зависят от времени. Этот случай называется \textbf{электростатикой} (или
\textbf{магнетостатикой}, в зависимости от того, поля какой природы
рассматриваются). 

В качестве аксиомы электростатики нам необходим какой-нибудь опытный
факт. Мы поступим немного нестандартным образом, а именно, возьмём за
основу факт, который именуется \textbf{теоремой Гаусса}. Возьмём
систему зарядов с полным зарядом $Q$ (это могут быть несколько
точечных зарядов, или заряд, размазанный по объёму). Окружим эту
систему зарядов поверхностью $S$. Тогда теорема Гаусса утверждает, что
поток электрического поля через эту поверхность равен

\begin{equation}
  \label{eq:theorem_gauss}
  \Phi = \int_S \vec{E} \cdot d\vec{S} = \frac{Q}{\epsilon_0}.
\end{equation}

Это достаточно общее утверждение, намного более общее, чем закон
Кулона (поэтому мы и взяли его за основу). В отличие от закона Кулона,
который не выполняется в случае электродинамики, теорема Гаусса
выполняется всегда. 

Кроме того, нам понадобится \textbf{принцип суперпозиции}: сила,
действующая на заряд, есть векторная сумма сил, действующих со стороны
прочих зарядов. По сути дела, это единственные аксиомы электростатики.

Из теоремы Гаусса получить закон Кулона очень просто. Возьмём точечный
заряд $q$, окружим его сферой радиуса $r$ и подсчитаем поток через эту
сферу. Ясно, что поле одного точечного заряда сферически симметрично,
поэтому на всех точках сферы оно будет одно и то же (поскольку зависит
лишь от расстояния до заряда). Раз так, то можно упростить интеграл
\eqref{eq:theorem_gauss}: 
\begin{equation}
  \label{eq:law_coulomb}
  \int_S E(r)\, dS = E(r) \int_S dS = E(r) 4\pi r^2 =
  \frac{q}{\eps_0}, \quad E(r) = k\frac{q}{r^2}.
\end{equation}

Это и есть \textbf{закон Кулона}: между двумя покоящимися зарядами
действует сила, пропорциональная зарядам и обратно пропорциональная
квадрату расстояния между ними (напомним, что сила равна $\vec{F} = q_0
\vec{E}$).

Таким образом, мы видим, что для статических точечных зарядов теорема
Гаусса эквивалентна закону Кулона. 

\subsection{Потенциал.}
\label{sec:potential}

Попробуем применить наши навыки векторного анализа к первому реальному
полю --- электрическому полю $\vec{E}$. Для упрощения формул (общий
результат всё равно останется таким же) предположим, что поле зависит
только от двух координат $x,y$ и не имеет составляющей по оси
$z$. Тогда

\begin{equation}
  \label{eq:rot_electrostatics_1}
  E_x = kq \frac{x}{(x^2+y^2)^{3/2}}, \quad   E_y = kq
  \frac{y}{(x^2+y^2)^{3/2}}, \quad E_z =0. 
\end{equation}

Заметим, что ротор нашего поля может иметь только компоненту по оси
$z$, так как две другие компоненты сразу обнуляются. По формуле
\eqref{eq:def_curl} получаем
\begin{equation}
  \label{eq:rot_electrostatics_2}
 \left( \rot \vec{E} \right)_x =0, \quad  \left( \rot \vec{E}
 \right)_y =0, \quad \left( \rot \vec{E} \right)_z = \pt_x E_y - \pt_y
 E_x = 0 \quad \Rightarrow \rot \vec{E} =0.
\end{equation}

Таким образом, мы выяснили, что электростатическое поле является
безвихревым. Таким образом, циркуляция этого поля по замкнутому
контуру равна нулю. Физический смысл этого довольно простой. 

Подставим наше поле в определение циркуляции \eqref{eq:curl}. Если
вдобавок мы домножим напряжённость $\vec{E}$ на какой-то пробный
заряд, то под интегралом в циркуляции будет стоять произведение силы
на элементарное перемещение, то есть, элементарная работа $dA$. Таким
образом, в данном случае циркуляция равна работе силы Кулона по
замкнутому контуру. 

Так как мы выяснили, что циркуляция нашего поля равна нулю, то
получается, что работа силы Кулона по замкнутому контуру равна
нулю. Это довольно важный результат. 

Кроме того, мы знаем (см. замечание после формулы
\eqref{eq:gauss_rot}), что любое поле, чей ротор равен нулю, можно
представить в виде градиента некоторой скалярной функции
$\phi$. Применим это к нашему случаю:

\begin{equation}
  \label{eq:def_potential}
  \vec{E}  = -\grad \phi = -\vn \phi.
\end{equation}

Знак минус в нашем определении стоит для удобства.

Функция $\phi$, чей градиент мы берём, называется \textbf{потенциалом}
электростатического поля $\vec{E}$. Заметим, кстати, что этот
потенциал определён с точностью до константы, т.к. градиент любой
постоянной функции равен нулю. 

Временно забудем про замкнутые контура, обратимся к незамкнутым
кривым. При перемещении нашего пробного заряда $q_0$ в электрическом поле
$\vec{E}$ должна совершаться какая-то работа $A$. Вычислим её. 

\begin{equation}
  \label{eq:work_statics_1}
  A = q_0 \int_C \vec{E} \cdot d\vec{l} = -q_0 \int_C \grad \phi\, d\vec{l}.
\end{equation}

Вспомним теперь теорему о градиенте \eqref{eq:def_curve_int_3}. Тогда
мы можем написать для последнего интеграла

\begin{equation}
  \label{eq:work_statics_2}
  A = -q_0 \int_C \grad \phi\, d\vec{l} = -q_0 (\phi_2 - \phi_1) = q_0
  (\phi_1 - \phi_2). 
\end{equation}

Итак, мы убедились, что работа, совершённая при перемещении пробного
заряда из точки 1 в точку 2 равна разности потенциалов в этих точках,
умноженной на величину заряда. 

Теперь было бы неплохо выяснить, как этот потенциал зависит от
расстояния до источника, то есть, до заряда. У нас есть соотношение
\eqref{eq:def_potential}, и в принципе его можно решить относительно
$\phi$. Мы же просто угадаем ответ:

\begin{equation}
  \label{eq:potential_r}
  \phi(\vec{r}\,) = \frac{kq}{r}. 
\end{equation}

Желающие могут проверить, что такой потенциал действительно
удовлетворяет соотношению \eqref{eq:def_potential}. 

Для потенциалов тоже действует принцип суперпозиции: потенциал системы
зарядов равен сумме потенциалов каждого заряда по отдельности. 

\subsection{Дивергенция.}
\label{sec:statics_div}

Теорему Гаусса можно переписать в более компактном и удобном виде
(перейти от интегралов к дифференциальным операциям). Именно, вспомним
теорему Гаусса--Остроградского \eqref{eq:theorem_gauss_ostograd} и
применим её к теореме просто Гаусса \eqref{eq:theorem_gauss}:

\begin{equation}
  \label{eq:statics_div_1}
  \int_V \div \vec{E} \cdot dV = \frac{Q}{\eps_0}.
\end{equation}

Итак, интеграл от дивергенции по объёму равен заряду (с точностью до
коэффициента). Что это означает? От чего ещё интеграл по объёму даёт
заряд? Разумеется, от объёмной плотности заряда (она вводится
абсолютно по аналогии с обычной плотностью). Тем самым, можно
написать, что 

\begin{equation}
  \label{eq:statics_div_2}
  \div \vec{E} = \frac{\rho}{\eps_0}.
\end{equation}

Это --- теорема Гаусса в альтернативной формулировке. Иногда она
удобнее, чем в исходном виде, потому что с её помощью можно эффективно
решить основную задачу электростатики --- по данному распределению
зарядов найти поля. 

Как это делается? Очень просто. Подставим определение потенциала
\eqref{eq:def_potential} в теорему Гаусса:

\begin{equation}
  \label{eq:poisson_1}
  \div \grad \phi =( \vn \cdot \vn) \phi =  -\frac{\rho}{\eps_0}. 
\end{equation}

Что такое оператор $\vn^2$, стоящий в скобках? Мы можем формально
умножить вектор $\vn$ сам на себя: 

\begin{equation}
  \label{eq:poisson_2}
  \vn \cdot \vn = \frac{\pt^2}{\pt x^2} + \frac{\pt^2}{\pt y^2} + \frac{\pt^2}{\pt z^2}.
\end{equation}

Оператор такого вида называется \textbf{лапласианом}, и иногда
обозначается буквой $\Delta$. Он дифференцирует два раза функцию по
каждой переменной и складывает результаты. Таким образом, мы можем
переписать уравнение для потенциала в форме

\begin{equation}
  \label{eq:poisson_3}
  \Delta \phi = -\frac{\rho}{\eps_0}.
\end{equation}

Это --- \textbf{уравнение Пуассона}. Оно связывает распределение
зарядов $\rho$ с потенциалом $\phi$. Иногда легче решить это
уравнение, получить потенциал и взять градиент, чтобы получить поле
$\vec{E}$. 

Теорему Гаусса можно использовать для доказательства ещё одного
фундаментального факта: в электрическом поле (при условии отсутствия
других сил) невозможно механическое равновесие. Ни в каком
электростатическом поле не существует точек устойчивого равновесия, за
исключением случая, когда заряды сидят друг на друге. 

Предположим, что такая точка $P$ существует. Рассмотрим положительный
заряд в точке $P$. Для того, чтобы заряд был в равновесии, нужно:

\begin{enumerate}
\item Поле в этой точке должно быть равно нулю;
\item Смещение заряда из $P$ в любую сторону должно вызывать силу,
  направленную против смещения;
\end{enumerate}

Из второго пункта следует, что вектор электрического поля в 
окрестности точки $P$ должен быть направлен в сторону точки
$P$. Однако, по теореме Гаусса это будет означать, что внутри
находится какой-то отрицательный заряд, что противоречит нашему
исходному предположению. 

Это утверждение верно и для более сложных конфигураций
зарядов. Однако, не стоит думать, что равновесие невозможно вообще:
если имеются дополнительные механические силы, то равновесие
возможно. Однако в чисто электростатическом поле равновесие
невозможно; этот факт называется \textbf{теоремой Ирншоу}. 

Например, именно поэтому невозможна модель атома, в которой электроны
неподвижны; именно поэтому Солнечная система до сих пор не развалилась
за счёт того, что планеты вращаются вокруг Солнца (теорема Ирншоу
применима к гравитационным силам тоже). Поэтому мы через некоторое
время перейдём к задачам электродинамики. 

\subsection{Энергия электростатического поля.}
\label{sec:statics_energy}

Мы рассмотрели электростатику с точки зрения сил взаимодействия между
заряженными телами. Однако, не хватает ещё энергетического подхода к
нашему взаимодействию. В частности, интересен такой вопрос: какая
энергия запасена в объёме $V$, в котором существует электрическое поле
$\vec{E}$? На этот вопрос мы и постараемся ответить.

Представим для начала что у нас имеются $N$ точечных зарядов
$q_i$. Тогда электростатическая энергия такой системы равна, очевидно, 

\begin{equation}
  \label{eq:energy_pointlike}
  U = \sum_{\text{все пары}} k \frac{q_i q_j}{|r_i - r_j|}.
\end{equation}

Если же у нас заряды распределены как-то по объёму, то сумма
заменяется интегралом. Очевидно, что энергия взаимодействия
элементарного заряда $\rho dV$ с потенциалом $\phi$, окружающим его
даётся выражением $\rho \phi dV$. Чтобы получить полную энергию,
надо проинтегрировать эту элементарную энергию по всему
объёму. Получим

\begin{equation}
  \label{eq:energy_volume}
  U = \frac12 \int \rho \phi\, dV.
\end{equation}

Коэффициент $1/2$ здесь взялся от того, что каждую пару из формулы
\eqref{eq:energy_pointlike} мы учли дважды. 

Здесь возникает интересный вопрос: где именно сосредоточена энергия?
Оказывается, можно показать, что энергия сосредоточена там, где
имеется электрическое поле. Зарядов при этом может там не быть: к
примеру, если распространяется электромагнитная волна, то она,
конечно, переносит с собой энергию, но не заряды. 

Таким образом, правильное описание энергии электростатического поля не
с помощью потенциалов и зарядов, а с помощью напряжённости
поля. Докажем сейчас, что энергия $U$ из формулы
\eqref{eq:energy_volume} может быть переписана в виде

\begin{equation}
  \label{eq:energy_field}
  U = \frac{\eps_0}{2} \int \vec{E} \cdot \vec{E} \, dV.
\end{equation}

Действительно, из уравнения Пуассона \eqref{eq:poisson_3} следует, что 

\begin{equation}
  \label{eq:energy_field_der_1}
  U = - \frac{\eps_0}{2} \int \phi \nabla^2 \phi \, dV.
\end{equation}

Выражение под интегралом можно преобразовать. 

\begin{equation}
  \label{eq:energy_field_der_2}
\phi \nabla^2 \phi = \phi \left( \frac{\pt^2 \phi}{\pt x^2} +
  \frac{\pt^2 \phi}{\pt y^2} +\frac{\pt^2 \phi}{\pt z^2} \right) =
\frac{\pt}{\pt x} \left( \phi \frac{\pt \phi}{\pt x} \right) - \left(
  \frac{\pt\phi}{\pt x} \right)^2 + \ldots = \vn \cdot (\phi \vn \phi)
- (\vn \phi) \cdot (\vn \phi).
\end{equation}

Тогда наш интеграл энергий примет вид

\begin{equation}
  \label{eq:energy_field_der_3}
  U = \frac{\eps_0}{2} \int (\vn \phi) (\vn \phi) \, dV -
  \frac{\eps_0}{2} \int \vn \cdot (\phi \vn \phi)\, dV.
\end{equation}

По теореме Гаусса--Остроградского второй интеграл можно преобразовать
в интеграл по поверхности: 

\begin{equation}
  \label{eq:energy_field_der_4}
   \int \vn \cdot (\phi \vn \phi)\, dV = \int (\phi \vn \phi) \cdot
   \vec{n}\, dS.
\end{equation}

В качестве поверхности возьмём сферу очень
большого радиуса $R$. Потенциал на большом расстоянии от любого
распределения зарядов ведёт себя как $1/R$, тогда градиент потенциала
(т.е. поле) ведёт себя как $1/R^2$. Тем самым подынтегральное
выражение ведёт себя как $1/R^3$. Площадь сферы же ведёт себя только
как $R^2$. Т.е. если наша сфера достаточно велика ($R \to \infty$), то
второй интеграл ведёт себя как $1/R$, и тем самым обращается в
ноль. Для энергии же получаем

\begin{equation}
  \label{eq:energy_field_der_5}
  U = \frac{\eps_0}{2} \int (\vn \phi) (\vn \phi) \, dV =
  \frac{\eps_0}{2} \int \vec{E} \cdot \vec{E} \, dV = \frac{\eps_0}{2}
  \int \vec{E}^2 \, dV.
\end{equation}

Таким образом, мы получили выражение для электростатической энергии
через электрическое поле в пространстве. 

\section{Электростатические аналогии. }
\label{sec:es_analogs}

Если две системы имеют одинаковые уравнения, то решения одной
системы можно использовать для решения другой. Сейчас мы рассмотрим
две ситуации, в которых наши знания из электростатики могут сильно
помочь. 

\subsection{Протекание тепла.}
\label{sec:heat}

Представим кусок какого-то материала, в котором температура меняется
от точки к точке. Возникает перемещение тепла, которое мы обозначим
$\vec{h}$. Это вектор, поскольку для потока тепла существенно
направление. Это --- тепловая энергия, которая проходит через
площадку, перпендикулярную вектору $\vec{h}$ за единицу времени. Каков
физический смысл величины $\div \vec{h}$? Это --- скорость ухода тепла
из данного места в единицу объёма. 

Представим, что наше вещество нагревается изнутри за счёт
какого-нибудь источника энергии, например, за счёт радиоактивного
изотопа мощностью $P$ (это энергия, производимая источником в единицу
времени). В силу того, что потоки тепла должны быть сбалансированы,
получаем 

\begin{equation}
  \label{eq:heat_1}
  \div \vec{h} = \vn \cdot \vec{h} = P.
\end{equation}

Кроме того, мы знаем, что вектор перемещения тепловой энергии зависит
от скорости изменения температуры на данном участке. Чем больше
разность температур, тем быстрее уходит энергия. Мы уже обсуждали, что
изменение температуры в пространстве удобно мерять градиентом $\grad
T$. Таким образом, из наших рассуждений получается, что 

\begin{equation}
  \label{eq:heat_2}
  \vec{h} = - K \vn T. 
\end{equation}

Коэффициент $K$ (коэффициент пропорциональности) называется
\textbf{коэффициентом теплопроводности} вещества. Склеивая эти два
уравнения вместе, мы получаем 

\begin{equation}
  \label{eq:heat_3}
  \vn \cdot (K \vn T) = -P.
\end{equation}

Это уравнение абсолютно эквиваленто уравнению Пуассона
\eqref{eq:poisson_3} с точностью до переобозначений. Это означает, что
если мы заменим $T$ на $\phi$ (а $\vec{h}$ на $\vec{E}$), то из
термодинамической системы мы попадём в электростатическую! Более того,
рассмотрим предельно простую ситуацию, когда источник тепла ---
точечный (или мы имеем точечный заряд). В этом случае мы знаем, что
потенциал спадает с расстоянием как $1/r$. Но то же самое можно
сказать о температуре! Таким образом, мы получили зависимость
температуры от расстояния пользуясь только лишь электростатической
аналогией и ничего не решая.


\begin{wrapfigure}{r}{40mm}
  \vspace{-1.2cm}
  \begin{center}
    \includegraphics[width=4cm,height=4cm]{heat.pdf}
  \end{center}
  \vspace{-0.7cm}
  \caption{Тепловой поток цилиндра.}
  \label{fig:heat}
\end{wrapfigure}


В теории теплоты также очень хорошо сработает закон Гаусса. Пусть
имеется цилиндр радиусом $r$, достаточно большой длиной $L$ и
постоянной температурой $T_1$, которая поддерживается за счёт
внутренних источников (пусть это будет нагретая провоолка внутри
цилиндра). Цилиндр окружён теплопроводящим материалом с постоянной
теплопроводностью $K$. Радиус этого внешнего слоя равен $R$. Чему
равна температура на его поверхности?

Будем решать эту задачу как электростатическую, по теореме Гаусса. Из
симметрии ясно, что температура зависит только от расстояния от центра
(также как потенциал зависит только от расстояния до центра
заряженного цилиндра). Окружим наш цилиндр поверхностью радиусом $r_1$
и длиной $L$. Площадь этой поверхности равна $S = 2\pi r_1 L$. Через
эту поверхность проходит количество тепла $G = 2\pi r_1 h L$ (мы учли,
что поле цилиндрически--симметрично) --- в электростатике, как мы
помним, поток поля пропорционален электрическому заряду. В то же
время, мы можем написать, что

\begin{equation}
  \label{eq:heated_cylinder_1}
  h = -K \frac{dT}{dr_1}.
\end{equation}

Опять-таки, из-за сферической симметрии от градиента остаётся только
лишь производная по радиусу. Это уравнение легко решить: 

\begin{equation}
  \label{eq:heated_cylinder_2}
  \frac{dT}{dr_1} = -\frac{G}{2\pi K L r_1} \quad \Rightarrow \quad T_2 - T_1 =
  -\frac{G}{2\pi K L} \log \frac{R}{r}.
\end{equation}

Чему отвечает такая система в электростатике? Видно, что разность
потенциалов зависит от расстояния до центра логарифмически. Также она
зависит от параметра $G$, который в электростатике имеет смысл
электрического заряда. Таким образом, можно сказать, что наш нагретый
цилиндр --- это цилиндрический конденсатор в теории электричества.

\subsection{Гидродинамика.}
\label{sec:hydro}

Вторая задача, решение которой может быть сведено к электростатике,
--- задача о постоянном течении несжимаемой жидкости. Дело в том, что
жидкость --- очень сложная для изучения среда и необходимы очень
серьёзные приближения, чтобы можно было что-то сосчитать. Поле,
которое нас интересует в гидродинамике --- поле скорости жидкости,
$\vec{v}$. 

Во-первых, будем считать, что в изучаемой нами области у жидкости нет
источников и стоков. Если движение жидкости постоянно, то скорость
$\vec{v}$ не зависит от времени. Пусть $\rho$ --- плотность жидкости,
тогда $\rho \vec{v}$ --- масса жидкости, проходящей за единичное время
через единичную площадку. Так как процессы рождения и исчезновения
жидкости отсутствуют, то дивергенция потока массы отсутствует, т.е.

\begin{equation}
  \label{eq:hydro_1}
  \vn \cdot (\rho \vec{v}) =0.
\end{equation}

Если предположить, что жидкость однородна и несжимаема, то плотность
постоянна, и её можно вынести за знак дивергенции. Таким образом, у
нас получается уравнение, напоминающее электростатику без зарядов: 

\begin{equation}
  \label{eq:hydro_2}
  \vn \cdot \vec{v} = 0.
\end{equation}

Чтобы добиться окончательного сходства с электростатикой, нужно
потребовать, чтобы течение жидкости было безвихревым. Этого можно
добиться при определённых условиях, на которых мы сейчас не будем
останавливаться. Условие безвихревого течения выглядит так: 

\begin{equation}
  \label{eq:hydro_3}
  \vn \times \vec{v} = 0.
\end{equation}

Эти два уравнения и будут основой для установления аналогии с
электростатикой. Мы уже видим, что наш вектор $\vec{v}$ играет здесь
ту же роль, какую вектор $\vec{E}$ играет в электричестве. Можно
заключить, что существует функция $\chi$, такая что

\begin{equation}
  \label{eq:hydro_4}
  \vec{v} = - \vn \chi.
\end{equation}


\begin{wrapfigure}{r}{40mm}
  \vspace{-1cm}
  \begin{center}
    \includegraphics[width=4cm,height=4cm]{fluid_lines.pdf}
  \end{center}
  \vspace{-0.7cm}
  \caption{Обтекание шара.}
  \label{fig:hydro}
\end{wrapfigure}


Рассмотрим теперь обтекание шара нашей медленной несжимаемой невязкой
жидкостью. Примерная картика потоков жидкости приведена на
рисунке. Наша задача будет состоять в определении зависимости скорости
$\vec{v}$ жидкости от расстояния от центра шара $r$. Сначала мы найдём
<<потенциал>> $\chi$, а потом возьмём от него градиент. Каким условиям
должен удовлетворять потенциал? Во-первых, течение должно
отсутствовать за поверхностью шара, это соответствует тому, что
производная потенциала по радиусу на границе шара должна обращаться в
ноль. Во-вторых, течение на бесконечности должно переходить в
константу. Эти два условия можно записать так:

\begin{equation}
  \label{eq:hydro_bc}
  \left.\frac{\pt \chi}{\pt r} = 0\right|_{r=a}, \quad \frac{\pt
    \chi}{\pt x} = v_0, \quad r \gg a.
\end{equation}

Наша задача, таким образом, эквивалента электростатической системе без
зарядов, но с однородным электрическим полем $\vec{E}_0$, в которую
помещена диэлектрическая сфера. Если бы сферы не было, то потенциал
всюду был бы равен $\phi = - E_0 z$. Однако, у нас есть сфера, которая
всё портит. 

Мы, однако, можем воспользоваться таким фактом: если мы угадаем
решение, которое удовлетворяет уравнению Пуассона и граничным
условиям \eqref{eq:hydro_bc}, то это решение будет
единственным. Поэтому будем искать решение в виде 

\begin{equation}
  \label{eq:hydro_4}
  \phi = -E_0 x + k\frac{px}{r^3}.
\end{equation}

Здесь $p$ --- параметр, который нам ещё предстоит определить. Заметим,
что этот потенциал на больших расстояних стремится к $-E_0 x$, как и
положено, согласно второму граничному условию. Второе слагаемое в
потенциале --- потенциал диполя, и поэтому он удовлетворяет уравнению
Пуассона. Теперь надо подобрать $p$. Выразим $x$ через $r$ и угол
$\theta$, получим

\begin{equation}
  \label{eq:hydro_5}
  \phi = - E_0 r \cos \theta + k \frac{p r \cos \theta}{r^3}.
\end{equation}

Теперь приравняем производную по $r$ к нулю, получим

\begin{equation}
  \label{eq:hydro_6}
  p = -2\pi \eps_0 a^3 E_0.
\end{equation}

Заметим, что если бы два слагаемых зависели от $\theta$ по-разному, мы
бы не смогли подобрать $p$. Это доказывает правильность нашего
первоначального решения искать $\phi$ именно в таком виде. Теперь
потенциал может быть записан так: 

\begin{equation}
  \label{eq:hydro_7}
  \phi = -E_0 \cos \theta \left(r + \frac{a^3}{2r^2} \right).
\end{equation}

Чтобы получить гидродинамический потенциал $\chi$, надо просто
поменять пару букв. 

\begin{equation}
  \label{eq:hydro_8}
  \chi = - v_0 \cos \theta \left(r + \frac{a^3}{2r^2} \right).
\end{equation}

Ответ для поля $\vec{v}$ приведён на картинке \ref{fig:hydro}.

\section{Уравнения Максвелла.}
\label{sec:maxwell}

Кроме электростатических сил, в поле наших интересов находятся также
магнитные статические силы. Как мы увидим, в чём-то они очень похожи
на электростатические, а в чём-то сильно отличаются. Наша задача
сейчас --- получить законы, описывающие магнитное поле, аналогичные
теореме Гаусса \eqref{eq:statics_div_2} и теореме о роторе
\eqref{eq:rot_electrostatics_2}. 

\subsection{Сохранение энергии в электродинамике.}
\label{sec:conservation_energy}

В случае электростатики отправной точкой у нас была теорема Гаусса;
сейчас мы применим соображения из закона сохранения энергии. Для
начала нам нужно само определение \textbf{магнитного поля}. Возьмём в
качестве пробного тела маленькую магнитную стрелку, которая может
свободно вращаться вокруг своего центра масс. Измерим величину момента
вращения; эта величина пропорциональна напряжённости магнитного поля
$\vec{H}$. За направление этого поля принимается направление, в
котором устанавливается эта стрелка (считая от южного полюса к
северному). По аналогии можно определить плотность энергии магнитного
поля как

\begin{equation}
  \label{eq:def_magnetic_energy}
  U_M = \frac{\mu_0}{2} \vec{H}^2,
\end{equation}
где $\mu_0$ играет роль <<магнитной проницаемости>>, по аналогии с
$\eps_0$. 

Попробуем теперь выяснить законы, описывающие электромагнитное
поле. Рассмотрим любую часть однородного, находящегося в покое тела, в
котором образовано электромагнитное поле. Рассмотрим процесс
перетекания энергии наружу (или приток снаружи). Введём \textbf{поток
  энергии} $\vec{S}$ --- величину, которая показывает, сколько энергии
проходит в единицу времени через единичную площадку.

Примем за аксиому такой факт: вектор потока энергии (иначе называемый
\textbf{вектором Пойнтинга}) пропорционален полям $\vec{E}, \vec{H}$ и
направлен по правилу правой руки, т.е.

\begin{equation}
  \label{eq:def_pointing}
  \vec{S} = \vec{E} \times \vec{H}.
\end{equation}

% Здесь $c$ --- коэффициент пропорциональности, точный смысл которого
% выяснится позднее. 

Теперь обсудим вопрос о превращении энергии. В каждой среде
электрическая энергия непрерывно переходит в тепловую, причём
количество перешедшей в тепло энергии пропорционально плотности
электрической энергии в данном месте:

\begin{equation}
  \label{eq:def_heat_from_electric}
  dt\, dV\, \frac{\eps_0}{2} \vec{E}^2 \cdot \const \equiv dt\, dV\,
  \sigma \vec{E}^2.
\end{equation}

Константа $\sigma$, стоящая в этом выражении, зависит исключительно от
природы вещества, в котором рассеивается электрическая энергия. Она
называется \textbf{проводимостью}. Для магнитной энергии аналогичного
явления не существует.

Вычислим происходящее за элемент времени $dt$ изменение
электромагнитной энергии в какой-либо части нашего тела. Для этого
всего лишь надо взять производную от выражения, которое даёт энергию. 

\begin{equation}
  \label{eq:conserv_energy_der_1}
  dt \int dV \left( \eps_0 (E_x \dot{E}_x + E_y \dot{E}_y +
    E_z \dot{E}_z) + \mu_0 (H_x \dot{H}_x + H_y \dot{H}_y +
    H_z \dot{H}_z) \right).
\end{equation}

Это изменение обсуловлено, во-первых, притоком энергии (вместе с
вектором Пойнтинга) в количестве

\begin{equation}
  \label{eq:conserv_energy_der_2}
  dt \int \vec{S} \cdot d\vec{A} =  \int dV\, \div \vec{S} ,
\end{equation}
и во-вторых, происшедшим за тот же интервал времени образованием
тепловой энергии в количестве 

\begin{equation}
  \label{eq:conserv_energy_der_3}
  dt \int dV\, \sigma \vec{E}^2.
\end{equation}

С точностью до знака мы можем написать уравнение баланса энергии: 

\begin{eqnarray}
  \label{eq:conserv_energy_der_4}
\nn
  \frac{\eps_0}{2} \left(E_x \dot{E}_x + E_y \dot{E}_y +
    E_z \dot{E}_z \right) +   \frac{\mu_0}{2} \left(H_x \dot{H}_x + H_y \dot{H}_y +
    H_z \dot{H}_z \right) + \\
+ \div \vec{S} + \sigma \left(E_x^2 + E_y^2 +E_z^2 \right) =0.
\end{eqnarray}

Уравнение \eqref{eq:conserv_energy_der_4} представляет собой уравнение
относительно шести компонент поля и их производных. Чтобы получить из
него уравнения, описывающие поля в отдельности, можно приравнять нулю
коэффициенты при этих компонентах. Например, приравнивая коэффициент
при $E_x$ нулю, получим

\begin{equation}
  \label{eq:conserv_energy_der_5}
  \eps_0 \dot{E}_x = \left( \frac{\pt H_z}{\pt y} - \frac{\pt H_y}{\pt
      z}\right) - \sigma E_x,
\end{equation}
а при $H_x$ 
\begin{equation}
  \label{eq:conserv_energy_der_5}
  \mu_0 \dot{H}_x = -\left( \frac{\pt E_z}{\pt y} - \frac{\pt
      E_y}{\pt z}\right).
\end{equation}

Очевидно, что оставшиеся четыре коэффициента дадут похожие уравнения;
их можно объединить в векторную форму: не расписывать всё по
компонентам, а написать сразу для векторов.

\begin{eqnarray}
  \label{eq:maxwell_half}
\nn
  \eps_0 \frac{\pt \vec{E}}{\pt t} &=& \rot \vec{H} - \sigma
  \vec{E},\\
  \mu_0 \frac{\pt \vec{H}}{\pt t} &=& -\rot \vec{E}.
\end{eqnarray}

Для ещё большего упрощения этой записи можно ввести вектора $\vec{D} =
\eps_0 \vec{E}$, $\vec{B} = \mu_0 \vec{H}$, а также вектор
\textbf{тока} $\vec{J} = \sigma \vec{E}$. Возьмём теперь дивергенцию
от второго уравнения Максвелла. Дивергенция ротора в правой части
всегда равна нулю, так что получаем

\begin{equation}
  \label{eq:gauss_magnet}
  \frac{\pt}{\pt t} \div \vec{B} = 0.
\end{equation}

Это означает, что если у нас с самого начала было немагнитное тело, то
оно не сможет никогда стать источником магнитного поля (иначе
дивергенция перестала бы быть равна нулю). Одиночных магнитных
источников в природе не существует (т.е. не существует магнитных
аналогов электрических зарядов). Таким образом, полная система
уравнений, описывающая электромагнитное поле (\textbf{уравнения
  Максвелла}), выглядит так:

\begin{eqnarray}
  \label{eq:maxwell_equations}
  \nn
  \rot \vec{E} &=& - \frac{\pt \vec{B}}{\pt t}, \qquad \div \vec{B} = 0,\\
  \rot \vec{H} &=& \vec{J} + \frac{\pt \vec{D}}{\pt t}, \quad \div
  \vec{D} = \rho.
\end{eqnarray}

\subsection{Согласованность уравнений Максвелла.}
\label{sec:cons_maxwell}

Разберём несколько следствий уравнений Максвелла. Во-первых,
рассмотрим самую простую ситуацию --- статические поля. Поле
называется статическим, если его конфигурация не меняется во времени;
это означает, что производные полей по времени равны нулю. В нашем
случае это даёт уравнения

\begin{equation}
  \label{eq:statics_maxwell_1}
  \rot \vec{E} = 0, \quad \rot \vec{H} = \vec{J}.
\end{equation}

Первое уравнение означает, что электростатическое поле не имеет
циркуляции. Мы уже знаем это свойство, так что сейчас мы просто
проверили согласованность всех наших выводов. Второе уравнение следует
разобрать подробнее. 

Для того, чтобы конфигурация была статической, необходимо, чтобы ни в
одной точке пространства не осуществлялась передача тепла. Таким
образом, есть ещё дополнительное условие: $\sigma \vec{E}^2 =
0$. Таким образом, либо проводимость $\sigma$ должна равняться нулю,
либо само поле $\vec{E}$ (из этого следует, кстати, что в статике в
проводнике нет электрического поля). В любом случае, ток тоже будет
равен нулю, так что получаем

\begin{equation}
  \label{eq:statics_maxwell_2}
  \rot \vec{H} = 0.
\end{equation}

Таким образом, магнетостатика и электростатика распадаются на два
совершенно независимых семейства уравнений, что позволяет говорить о
наложении электрических и магнитных полей. 

\section{Магнетостатика.}
\label{sec:magnetostatics}

Сосредоточимся немного подробнее теперь на магнитных полях. Случай
электростатики разобран довольно-таки подробно, и есть надежда ожидать
что-то похожее для магнитных полей. 

\subsection{Закон Био--Савара--Лапласа.}
\label{sec:biot-savart-laplace}

Посмотрим теперь, как будут выглядеть магнитные поля в стационарном
случае (т.е. когда производные полей по времени равны нулю, но ток
может протекать). Для начала введём \textbf{векторный потенциал}
$\vec{A}$. Заметим, что так как $\div \vec{B} = 0$, то существует
такой вектор, что 

\begin{equation}
  \label{eq:def_vector_potential}
  \vec{B} = \rot \vec{A}.
\end{equation}

Подставляя это соотношение в уравнения Максвелла
\eqref{eq:maxwell_equations}, получаем 

\begin{equation}
  \label{eq:laplace_vector_pot_1}
  \rot \rot \vec{A} = \mu_0 \vec{J}.
\end{equation}

Заметим теперь, что вектор $\vec{A}$ определён неоднозначно: мы можем
добавить к нему градиент любого скаляра, и ничего не изменится, так
как ротор любого градиента равен нулю (это называется
\textbf{калибровочной инвариантностью}). Чтобы всё-таки как-то
зафиксировать вектор $\vec{A}$, можно наложить на него, например,
такое условие: $\div \vec{A} = 0$.

Далее, можно преобразовать уравнение
\eqref{eq:laplace_vector_pot_1}. Заметим для этого, что $\rot \rot
\vec{A} = \grad \div \vec{A} - \Delta \vec{A}$. Таким образом,
получаем для нашего вектор--потенциала: 

\begin{equation}
  \label{eq:laplace_vector_pot_2}
  \Delta \vec{A} = - \mu_0 \vec{J}. 
\end{equation}

Заметим, что это уравнение очень похоже на уравнение Пуассона для
скалярного потенциала \eqref{eq:poisson_3}. А раз мы знаем решение
уравнение Пуассона для $\phi$, то можем легко по аналогии написать
решение уравнения для $\vec{A}$: 

\begin{equation}
  \label{eq:solution_vector_pot}
  \vec{A}(\vec{r}) = \frac{\mu_0}{4\pi} \int \frac{\vec{J}(\vec{r}')}{r} dV'.
\end{equation}

Это выражение --- точная аналогия выражения для потенциала от
произвольной системы зарядов. Теперь мы можем получить выражение и для
напряжённости магнитного поля (\textbf{закон
  Био--Савара--Лапласа}). 

Введём вместо плотности тока $\vec{J}$ ток $I$ по следующему правилу:
ток показывает, какой заряд протёк через площадку площади $S$ и
направлением $\vec{S}$.

Тогда наш интеграл перепишется в виде

\begin{equation}
  \label{eq:der_bio-savart-laplace}
  \vec{A} = \frac{\mu_0}{4\pi} \int \frac{I \cdot d\vec{l}'}{r}.
\end{equation}

Подставим теперь наш потенциал в формулу
\eqref{eq:def_vector_potential}. Получим

\begin{equation}
  \label{eq:bio-savart-laplace_1}
  \vec{H} =\frac{\mu_0}{4\pi} \int I\, \vn \times \left( \frac{1}{r} \cdot d\vec{l}' \right).
\end{equation}

Можно показать, что если у вас есть скалярная функция $\phi$ и
векторная $\vec{F}$, то

\begin{equation}
  \label{eq:rot_prop}
  \vn \times (\phi \vec{F}) = \vn \phi \times \vec{F} + \phi \cdot \vn
  \times \vec{F}.
\end{equation}

Здесь надо быть аккуратным: в нашем случае оператор $\vn$ действует на нештрихованные
координаты (напомним, что это комбинация производных), и поэтому его
действие на вектор $d\vec{l}'$ нулевое. Таким образом, от двух
слагаемых останется только одно: 

\begin{equation}
  \label{eq:der_bio-savart-laplace_2}
  \vec{H} = \frac{\mu_0}{4\pi} \int I \vn \frac{1}{r} \times d\vec{l}'.
\end{equation}

Как вычислить градиент такой функции? Мы уже знаем
ответ. Действительно, $1/r$ --- потенциал одиночного точечного
заряда. Градиент этого потенциала --- электрическое поле. По закону
Кулона, оно направлено по радиус--вектору и убывает как $1/r^2$. Таким
образом, 

\begin{equation}
  \label{eq:der_bio-savart-laplace_3}
  \vec{H} = \frac{\mu_0}{4\pi} \int \frac{I \cdot d\vec{l}' \times \vec{r}}{r^3}.
\end{equation}

Это и есть закон Био--Савара--Лапласа. Он позволяет вычислить
магнитное поле от любого куска провода с током. Физический смысл его
примерно такой же, как у закона Кулона в электростатике: при данном
распределении тока существует правило вычисления напряжённости
магнитного поля. 

\subsection{Закон Ампера.}
\label{sec:amperes_law}

Выводя закон Био--Савара--Лапласа, мы заметили ещё одно соотношение,
непосредственно следующее из уравнений Максвелла в стационарном
пределе. Именно, 

\begin{equation}
  \label{eq:der_amperes_law}
  \rot \vec{B} = \mu_u \vec{J}.
\end{equation}

Вспомним, что по теореме Стокса \eqref{eq:theorem_curl_5_stokes} это
соотношение можно переписать так: 

\begin{equation}
  \label{eq:amperes_law}
  \oint \vec{B} \cdot d \vec{l} = \mu_0 \int \vec{J} \cdot d\vec{S} \equiv
  \mu_0 \cdot I.
\end{equation}

То есть, если мы вычислим циркуляцию магнитного поля $\vec{B}$ по
замкнутому контуру $C$, то она будет пропорциональна потоку плотности
тока через поверхность, охватываемую этим контуром. Это ---
\textbf{закон Ампера}. Нетрудно заметить, что он очень напоминает
теорему Гаусса для электростатики. Более того, в предыдущем разделе мы
видели, что он играет ту же роль, что и теорема Гаусса, при выводе
закона БСЛ (при выводе закона Кулона мы тоже исходили из теоремы
Гаусса). 

\subsection{Примеры.}
\label{sec:magnetostatics_examples}

С помощью закона Ампера \eqref{eq:amperes_law} и Био--Савара--Лапласа
\eqref{eq:der_bio-savart-laplace_3} можно легко вычислять магнитные
поля для простых распределений токов. 

\subsubsection{Поле провода. }
\label{sec:mfield_wire}

Возьмём для примера длинный тонкий провод, по которому течёт ток $I$,
и вычислим магнитное поле снаружи такого провода.

Проще всего воспользоваться законом Ампера. Из симметрии задачи ясно,
что поле $\vec{B}$ будет обладать цилиндрической симметрией, поэтому в
качестве контура для закона Ампера возьмём окружность радиусом $r$ с
центром в центре провода. Циркуляция магнитного поля будет просто
равна

\begin{equation}
  \label{eq:mfield_thin_current_1}
  \oint \vec{B} \cdot d \vec{l} = B \cdot 2 \pi r.
\end{equation}

По закону Ампера эта циркуляция равна просто $\mu_o I$, так что для
нашего магнитного поля просто получаем

\begin{equation}
  \label{eq:mfield_thin_current_2}
  B =  \frac{\mu_0 I}{2\pi r}.
\end{equation}

\subsubsection{Поле соленоида.}
\label{sec:mfield_solenoid}

Теперь возьмём очень длинный провод, свёрнутый в спираль ---
соленоид. Можно показать (хотя это немного муторно), что магнитное
поле снаружи соленоида равно нулю. Далее, по симметрии очевидно, что
внутри соленоида оно направлено по оси соленоида. 

Возьмём прямоугольный контур $C$ ширины $L$, который охватывает $N$
кусочков провода. Вычислим циркуляцию поля $\vec{B}$ вдоль этого
контура --- по закону Ампера она связана с током, который пронизывает
этот контур. 

\begin{equation}
  \label{eq:der_mfield_solenoid}
  B \cdot l = \mu_o N I.
\end{equation}

Вводя плотность витков по формуле $n = N/L$, получим для величины
магнитного поля

\begin{equation}
  \label{eq:mfield_solenoid}
  B = \mu_0 n I.
\end{equation}

Согласно одному из уравнений Максвелла, магнитные линии нигде не
начинаются и не заканчиваются. В случае соленоида это эквивалентно
тому, что они выходят из одного его конца и входят в другой
непрерывным образом. У магнитных линий в самом деле нет источника. 

\subsection{Энергия взаимодействия тока с магнитным полем. }
\label{sec:magnet_energy}

В случае электростатики мы видели, что заряд, помещённый в
электрическое поле, является источником потенциальной энергии
\eqref{eq:energy_volume}. Нечто аналогичное должно быть и в случае
магнитного поля. 

Напомним, что компоненты потенциала $\vec{A}$ подчиняются тому же
уравнению, что и потенциал $\phi$. Отсюда можно сделать вывод об их
сходстве и по аналогии написать выражение для энергии взаимодействия
проводника с плотностью тока $\vec{J}$ с магнитным полем: 

\begin{equation}
  \label{eq:magnet_energy_1}
  U = - \int \vec{J} \cdot \vec{A} \, dV.
\end{equation}

Здесь плотность тока $\vec{J}$ играет роль плотности зарядов $\rho$, а
потенциал $\vec{A}$ играет роль потенциала $\phi$. 

Вспомним теперь, что плотность тока можно переписать через
ток. Получаем

\begin{equation}
  \label{eq:magnet_energy_2}
  U = - I \int \vec{A} \cdot d \vec{r} = - I \int \vec{B} \cdot d \vec{S}.
\end{equation}

\section{Физика индукции.}
\label{sec:induction}

\subsection{Закон Фарадея.}
\label{sec:faradays_law}

Будем постепенно усложнять задачу и включим в наше рассмотрение
динамику. Именно, для начала рассмотрим воздействие переменного
магнитного поля на электрические поля. Согласно одному из уравнений
Максвелла, 

\begin{equation}
  \label{eq:der_faraday_1}
  \rot \vec{E} = -\frac{\pt \vec{B}}{\pt t}.
\end{equation}

Применим к этому равенству теорему Стокса
\eqref{eq:theorem_curl_5_stokes}:

\begin{equation}
  \label{eq:der_faraday_2}
  \oint_{\Gamma} \vec{E} \cdot d \vec{l} = - \int_S \frac{\pt
    \vec{B}}{\pt t} \cdot \vec{n}\, dS.
\end{equation}

Предположим теперь, что наша поверхность $S$ фиксирована. Тогда
производную по времени можно вынести за знак интеграла:

\begin{equation}
  \label{eq:der_faraday_3}
  \oint_{\Gamma} \vec{E} \cdot d \vec{l} = - \frac{\pt}{\pt t} \int_S
  \vec{B} \cdot \vec{n}\, dS.
\end{equation}

Интеграл, стоящий в правой части --- поток магнитного поля $\Phi$ (по
определению). Интеграл, стоящий в левой части называется
\textbf{электродвижущей силой} $\eps$ (заметим, что по смыслу это не сила,
т.к. имеет размерность вольт). Также отметим, что кривая $\Gamma$
может быть произвольной, не обязательно привязанной к какому-то
проводнику. Это --- просто математический объект. 

Записывая компактно наш результат, получим

\begin{equation}
  \label{eq:faradays_law}
  \eps = - \frac{\pt \Phi}{\pt t}.
\end{equation}

Это соотношение выражает \textbf{закон Фарадея}. Он гласит, что при
изменении потока магнитного поля через какую-либо поверхность на
границе этой поверхности возникает ЭДС.

\subsection{Сила Ампера.}
\label{sec:amperes_force}

Для дальнейшего нам необходимо выяснить, с какой силой действует
магнитное поле $\vec{B}$ на кусок проводника, по которому течёт ток
$I$. Чтобы найти механическую силу $\vec{F}$, которая действует на
кусок проводника с током, мы воспользуемся методом \textit{виртуальных
  перемещений}. Сдвинем наш проводник с током на $\delta \vec{r}$ и
посмотрим, какую работу при этом совершит сила. Очевидно, эту работу
можно записать в виде $\sum \vec{F} \cdot \delta \vec{r}$. Она равна
убыли потенциала (см. \eqref{eq:magnet_energy_2}):

\begin{equation}
  \label{eq:der_amperes_force_1}
 \sum \vec{F} \cdot \delta \vec{r} = -\delta U = I \delta \left( \int \vec{B}  \cdot d\vec{S}\right). 
\end{equation}

\begin{wrapfigure}{r}{40mm}
  \vspace{-1.5cm}
  \begin{center}
  \includegraphics[width=4cm]{ampere.pdf}  
  \end{center}
  \vspace{-1cm}
  \caption{К вычислению силы Ампера.}
  \label{fig:amperes_force}
  \vspace{-1.1cm}
\end{wrapfigure}

Фактически, нам нужно вычислить изменение потока. Так как магнитное
поле постоянно, изменение потока складывается лишь за счёт изменения
площади поверности, которая пронизывается магнитным полем. Если мы
рассматриваем кусочек проводника длиной $d\vec{r}$, то при его сдвиге
на $\delta \vec{r}$ заметается площадь 

\begin{equation}
  \label{eq:der_amperes_force_2}
  d\vec{S} = \delta \vec{r} \times d \vec{r}.
\end{equation}

Подставляя это в выражение для изменения потенциала, получаем

\begin{equation}
  \label{eq:der_amperes_force_3}
  -\delta U = I \int \vec{B} \cdot \left( \delta  \vec{r}
    \times d \vec{r}\right) = I \int \delta \vec{r} \cdot (d\vec{r} \times \vec{B}).
\end{equation}

Отсюда немедленно следует, что наша сила равна 

\begin{equation}
  \label{eq:amperes_force}
  \vec{F} = I  \int d\vec{r} \times \vec{B}.
\end{equation}

Она называется \textbf{силой Ампера} и показывает, с какой силой
постоянное магнитное поле действует на проводник с током $I$. Можно
заметить, что эта сила направлена перпендикулярно магнитному полю.

Заметим также, что эта сила обеспечивает взаимодействие между двумя
проводами с током. Действительно, ток создаёт магнитное поле вокруг
себя (по закону Ампера \eqref{eq:amperes_law}). Это магнитное поле
действует на второй провод с током по уравнению
\eqref{eq:amperes_force}. 

\begin{wrapfigure}{r}{40mm}
  \vspace{-1cm}
  \begin{center}
  \includegraphics[width=4cm]{2wires.pdf}  
  \end{center}
  \vspace{-1cm}
  \caption{Притяжение двух проводов.}
  \label{fig:2wires}
\end{wrapfigure}

Разберём простейшую ситуацию, когда провода параллельны друг другу, а
токи в них сонаправлены. Вокруг первого тока создаётся магнитное поле,
закрученное вокруг провода по часовой стрелке (если смотреть по
направлению тока). Это поле действует на участок тока $d\vec{r}$
второго провода. Сила Ампера --- векторное произведение, поэтому её
направление можно получить по правилу правой руки. Отсюда видно, что
сила Ампера $\vec{F}$ направлена перпендикулярно второму проводу по
направлению к первому. Очевидно, аналогичное рассуждение можно
провести и для силы, действующей на первый провод со стороны второго.

Из этого рассмотрения видно, что противоположно направленные токи
всегда отталкиваются. 

\subsection{Сила Лоренца.}
\label{sec:lorentz_force}

Сила Ампера позволяет довольно просто установить ещё один общий закон,
универсальный для всей электродинамики. Действительно, зададимся
вопросом: что такое ток в проводе? Это не что иное, как упорядоченный
поток электронов. Представим, что у нас имеется кусочек провода длины
$d\vec{l}$, и площадью сечения $S$. На него действует сила Ампера,
равная
\begin{equation}
  \label{eq:der_lorentz_force_1}
  d\vec{F} = I \cdot d\vec{l} \times \vec{B} = \vec{J} \times \vec{B}\, dV.
\end{equation}

Но что такое $J \, dV$? Это просто количество заряда, протёкшее через
сечение нашего проводника. Если концентрация электронов равна $n$, заряд
каждого из них $q$, а скорость $\vec{v}$, то 

\begin{equation}
  \label{eq:der_lorentz_force_2}
  \vec{J} = n q \vec{v}.
\end{equation}

Подставляя эту плотность тока в силу Ампера, получим

\begin{equation}
  \label{eq:der_lorentz_force_3}
  d\vec{F} = q \, \vec{v} \times \vec{B} \, n\, dV.
\end{equation}

Произведение $n \, dV$ даёт просто количество электронов в объёме
$dV$. Таким образом, можно сказать, что сила Ампера --- это сила,
действующая на одиночный заряд, умноженная на количество
зарядов. Получается, что сила со стороны магнитного поля, действующая
на одиночный заряд, равна

\begin{equation}
  \label{eq:lorentz_force}
  \vec{F} = q \vec{v} \times \vec{B}.
\end{equation}

Это --- \textbf{сила Лоренца}. Опять же, она перпендикулярна скорости
зарядов, а также магнитному полю, которое действует на них. Из этого
ясно, что сила Лоренца никогда не совершает работы (потому что её
мощность равна нулю). 

Если теперь вспомнить, что на заряженную частицу может ещё действовать
сила со стороны электрического поля, то получится общее выражение для
силы в электродинамике:

\begin{equation}
  \label{eq:force_electrodynamics}
  \vec{F} = q (\vec{E} + \vec{v} \times \vec{B}).
\end{equation}

Имея на руках выражение для силы Лоренца, мы можем понять, как
меняется ЭДС в случае изменения магнитного потока не за счёт изменения
магнитного поля, а за счёт изменения площади контура. Напомним, в
выводе закона Фарадея \eqref{eq:faradays_law} мы предполагали, что наш
контур остаётся неизменным, а меняется лишь магнитное поле. 

\begin{wrapfigure}{r}{40mm}
  \vspace{-1cm}
  \begin{center}
  \includegraphics[width=4cm]{faraday.pdf}  
  \end{center}
  \vspace{-1cm}
  \caption{Изменение потока.}
  \label{fig:faraday}
\end{wrapfigure}

Рассмотрим теперь противоположную ситуацию. Пусть у нас есть
проволочная U-образная петля шириной $h$, по которой может скользить без трения
перемычка. Пусть имеется магнитное поле $\vec{B}$, перпендикулярное
плоскости петли. Заставим перемычку перемещаться со скоростью
$V$. Очевидно, что на заряды в перемычке будет действовать сила $F =
V B$ (мы рассматриваем единичные заряды), направленная вдоль
перемычки. В остальных местах перемычки никакой силы действовать не
будет. Таким образом, циркуляция вдоль замкнутого контура равна
$\eps=-hVB$ (мы учли направление обхода). 

Заметим, что то же самое мы получили бы, если бы посчитали изменение
потока по правилу Фарадея. Действительно, поток зависит от положения
перемычки так: $\Phi = h B L$. При этом $L=Vt$, так как перемычка
движется равномерно. Получаем из \eqref{eq:faradays_law}, что $\eps =
-h VB$.

Аналогичное рассуждение можно провести для контура произвольной
формы. Таким образом, можно показать, что возникающая в контуре ЭДС
равна изменению потока по времени вне зависимости от того, чем вызвано
это изменение --- магнитным полем, или изменяющимся контуром. 

\subsection{Физический смысл добавки Максвелла. }
\label{sec:physical_sense_maxwell}

В уравнении Максвелла для ротора магнитного поля мы сталкиваемся с
тем, чего ещё не было в нашем опыте: магнитное поле создаётся как
изменением магнитного поля, так и током. Почему нельзя было
ограничиться чем-то одним, как в случае электрического поля? 

Ответ на этот вопрос даёт один несложный пример. Рассмотрим плоский
конденсатор, заряжающийся от провода. Пусть заряд $Q$ на пластинах
изменяется. Через провод таким образом будет течь ток $dQ/dt$. Этот
ток по закону Ампера будет создавать магнитное поле (мы уже вычислили
его в разделе \ref{sec:mfield_wire}). Здесь мы рассмотрили контур $C$,
расположенный вокруг провода. При этом можно вспомнить, что снаружи
проводника поля нет, поэтому добавка $\pt \vec{E}/\pt t$ никак не
меняет ситуацию.

\begin{wrapfigure}{r}{40mm}
  \vspace{-1cm}
  \begin{center}
  \includegraphics[width=4cm]{displacement.pdf}  
  \end{center}
  \vspace{-1cm}
  \caption{Магнитное поле возле конденсатора.}
  \label{fig:displacement}
\end{wrapfigure}


Теперь рассмотрим контур $C'$, расположенный между пластинами. Никакого
тока через поверхность, ограниченную таким контуром не
проходит. Означает ли это, что никакого магнитного поля нет? Это было
бы очень странно. И действительно, теперь в дело вступает добавка из
уравнения Максвелла. Силовые линии направлены от одной пластины к
другой, поэтому поток электрического поля по теореме Гаусса равен
заряду на пластине $Q$. А циркуляция магнитного поле, как можно
заметить, равно скорости изменения этого потока (по тем же причинам,
что циркуляция электрического поля равна скорости изменения магнитного
потока). 

Циркуляция магнитного поля по контуру радиуса $r$ равна $2\pi r
\mu_0 B$. Скорость изменения потока равна 

\begin{equation}
  \label{eq:phys_sense_maxwell_1}
  \frac{1}{\eps_0} \frac{dQ}{dt}.
\end{equation}

Таким образом, мы получаем тот же ответ, что и для магнитного поля
вокруг провода \eqref{eq:mfield_thin_current_2} (т.к. $\vec{D} =
\eps_0 \vec{E}$). Добавка Максвелла обеспечивает согласованность нашей
картины.

\section{Движущееся электромагнитное поле. }
\label{sec:moving_plate}

\subsection{Скорость света.}
\label{sec:speed_of_light}

Рассмотрим неподвижный заряженный лист, размещённый в плоскости
$yz$. Пусть он быстро приобретает скорость $u$ в направлении оси $y$ и
движется с этой постоянной скоростью. Итого, мы внезапно получаем ток
плотностью $J$ (ток, отнесённый к единице ширины по оси
$z$). Рассмотрим также неподвижный противоположно заряженный лист в
той же плоскости $yz$, чтобы забыть про электростатические эффекты.

Если имеется лист с током в положительном $y$ направлении, то
возникнет магнитное поле, направленное в отрицательном $z$
направлении (при $x>0$) и в положительном $z$ направлении при $x<0$. 

Ясно, что если магнитное поле скачком меняется от нуля до конечной
величины, то появиляются очень большие электрические эффекты. Так что
появляется меняющееся магнитное поле и меняющееся электрическое. Таким
образом, возникает производная $\pt \vec{E} / \pt t$, которая вместе с
током $\vec{J}$ будет вносить вклад в магнитное поле (по уравнению
Максвелла). Выходит, что уравнения очень сильно зацеплены друг с
другом и сильно зависят от времени. 

Попробуем, однако, описать количественно, что происходит. Посмотрим на
эту систему сверху. Ток будет направлен на нас, магнитное поле лежит
в плоскости рисунка. Если смотреть сбоку, то лист будет двигаться
вверх, а поле будет смотреть на нас или от нас. 

Предположим такое распределение полей. Пусть с момента начала движения
нашей плоскости прошло время $t$. Тогда утверждается, что поля
$\vec{E}, \vec{B}$ будут существовать в пространстве только до
$x=vt$, где $v$---некоторая константа. Проверим, как это согласуется с
уравнениеями Максвелла. 

\begin{figure}[h]
  \centering
  \subfloat[Вид сбоку.]{\label{fig:plate2}\includegraphics[scale=1]{plate2.pdf}}
  \hspace{2cm}
  \subfloat[Вид сверху.]{\label{fig:plate1}\includegraphics[scale=1]{plate1.pdf}}
  \caption{Движущаяся заряженная плоскость.}
  \label{fig:moving_plate}
\end{figure}

Проведём прямоугольный контур $\Gamma_2$ (см. рисунок \ref{fig:plate2}). Он
охватывает часть пространства, где есть поля (слева от прямой $x=vt$),
и часть пространства, где нет полей. Если фронт движется со скоростью
$v$, то поток магнитного поля через $\Gamma_2$ будет меняться тоже со
скоростью $v$. Если ширина прямоугольника равна $L$, то наводящаяся по
контуру ЭДС равна, по закону Фарадея,

\begin{equation}
  \label{eq:moving_plate_1}
  E = v B.
\end{equation}

То есть, если отношение $E$ к $B$ равно $v$, то наши поля будут
удовлетворять закону Фарадея (т.е. одному из уравнений Максвелла). 

Однако, у нас есть ещё одно уравнение: оно связывает ротор магнитного
поля с изменением электрического и током. Чтобы применить его,
посмотрим на нашу систему сверху. Опять нарисуем прямоугольный контурн
$\Gamma_1$ (см. рисунок \ref{fig:plate1}), который пересекает волновой
фронт. Токов через этот контур не проходит, поэтому циркуляция $B$
равна скорости изменения потока $E$. Аналогично предыдущим
соображениям, находим, что

\begin{equation}
  \label{eq:moving_plate_2}
  E = \frac{1}{\eps_0 \mu_0 v} B.
\end{equation}

Собирая вместе уравнения \eqref{eq:moving_plate_1} и
\eqref{eq:moving_plate_2}, получаем, что скорость волнового фронта
равна 

\begin{equation}
  \label{eq:velocity_front}
  v = \frac{1}{\sqrt{\eps_0 \mu_0}}.
\end{equation}

Таким образом, волновой фронт распространяется с \textit{конечной
  скоростью}, которая может быть вычислена на основании электрических
и магнитных свойств среды (в данном случае --- вакуума). 

Обе константы $\eps_0$ и $\mu_0$ могут быть измерены
экспериментально: нужно лишь померять силы взаимодействия зарядов или
токов. На основании этих данных можно вычислить скорость
$v$. Выясняется, что она с огромной точностью равна скорости света
$c$. Разумеется, это не простое совпадение. 

Кроме того, можно выяснить, что особенности нашего решения довольно
универсальны: в любой электромагнитной волне электрическое поле
перпендикулярно фронту волны, а магнитное поле перпендикулярно
электрическому. Модуль напряжённости электрического поля $E$ всегда
равен $cB$, где $c$---скорость распространения фронта волны, она же
скорость света. 

\subsection{Электромагнитная волна. }
\label{sec:em_wave}

В предыдущем разделе мы видели, что ток создаёт переменное
электромагнитное поле во всём пространстве. Означает ли это, что волны
могут существовать только тогда, когда есть ток? Не совсем.

У нас был ток, равный нулю, а потом мы его внезапно включали. Что
случится, если мы его столь же внезапно выключим? Осуществить эту
процедуру можно с помощью принципа суперпозиции: возьмём второй лист,
заряженный противоположно и через время $T$ после начала движения
первого листа, будем двигать его в противоположном направлении с той
же скоростью. Полный ток от двух листов, таким образом, сначала будет
равен нулю, потом $J$, потом через время $T$ он снова равен нулю. Мы
получаем прямоугольный импульс тока. 

Новый отрицательный ток создаёт такие же поля, как и первый, только
направленные в противоположном направлении и, разумеется, с
запаздыванием на $T$. Волновой фронт от этого движения тоже двигается
со скоростью $c$. В момент времени $t$ он достигает расстояния $x = c
(t-T)$. 

Суммируя вклады от обоих токов, мы получим, что при $x>ct$ поля равны
нулю (за пределами волнового фронта), между $x=c(t-T)$ и $x=ct$ они
постоянны, а для $x<c(t-T)$ они снова равны нулю. 

Иными словами, мы получили кусочек поля толщиной $cT$, который
передвигается сквозь пространство безо всяких токов. 

Что позволяет этому кусочку поля существовать? Разумеется,
динамические части уравнений Максвелла. Предположим, что магнитное
поле исчезло бы. Тогда появилось меняющееся магнитное поле, которое
должно было вызвать вихревое электрическое. Если бы это электрическое
поле попыталось исчезнуть, то его изменение породило бы магнитное. То
есть, энергия электромагнитной волны всё время перекачивается из
одного поля в другое, при этом не исчезая. 

\section{Волновые процессы.}
\label{sec:waves}

Настало время соединить все уравнения Максвелла воедино. Проделывая
это, мы обнаружим ещё один эффект, который запрятан в них. 

\subsection{Калибровочная инвариантность.}
\label{gauge}

Напомним, что решение уравнения $\div \vec{B} = 0$ выглядит так: 

\begin{equation}
  \label{eq:gauge_1}
  \vec{B} = \vn \times \vec{A}.
\end{equation}

Подставим это в закон Фарадея:

\begin{equation}
  \label{eq:gauge_2}
  \vn \times \vec{E} = - \frac{\pt}{\pt t} \vn \times \vec{A}
  \Rightarrow \vn \times \left( \vec{E} + \frac{\pt \vec{A}}{\pt t}
  \right) = 0.
\end{equation}

Но мы уже неоднократно замечали, что если ротор чего-то равен нулю, то
это что-то --- градиент некоторой скалярной функции $-\phi$:

\begin{equation}
  \label{eq:gauge_3}
  \vec{E} = - \vn \phi - \frac{\pt \vec{A}}{\pt t}.
\end{equation}

Фактически, это переформулированный закон Фарадея. Мы видим, кстати,
что для описания электромагнитных полей нужны всего четыре функции:
скалярный потенциал $\phi$ и вектор--потенциал $\vec{A}$ (который,
разумеется, содержит четыре компоненты). 

Кстати, из уравнения \eqref{eq:gauge_1} видно, что если к потенциалу
$\vec{A}$ добавить градиент какого--нибудь поля $\psi$, то магнитное
поле не изменится, так как ротор любого градиента равен нулю. А что
произойдёт с электрическим? Как видно из формулы \eqref{eq:gauge_3},
оно поменяется, если не принять особых мер. 

Примем же эти меры. Если вдобавок мы будем менять скалярный потенциал,
то всё будет хорошо: и магнитное, и электрическое поле останутся
прежними при следующих заменах:

\begin{equation}
  \label{eq:gauge_transform}
  \vec{A} \to \vec{A} + \vn \psi, \quad \phi \to \phi - \frac{\pt
    \psi}{\pt t}.
\end{equation}

Эти изменения в потенциалах называются \textbf{калибровочными
  преобразованиями}. Говорят, что электродинамика \textbf{инвариантна}
относительно калибровочных преобразований, потому что физические поля
$\vec{E}$ и $\vec{B}$ не зависят от выбора функции $\psi$. 

Продолжим жонглировать уравнениями Максвелла. Подставим уравнение
\eqref{eq:gauge_2} в теорему Гаусса:

\begin{equation}
  \label{eq:gauge_4}
  \vn \cdot \vec{E} = \frac{\rho}{\eps_0} \Rightarrow \vn \cdot \left(
   -\vn \phi - \frac{\pt \vec{A}}{\pt t} \right) =
 \frac{\rho}{\eps_0} \Rightarrow -\vn^2 \phi - \frac{\pt}{\pt t} \vn
 \cdot \vec{A} = \frac{\rho}{\eps_0}.
\end{equation}

Это первое уравнение, связывающее потенциалы с источниками. Чтобы
получить второе уравнение, надо подставить \eqref{eq:gauge_1} и
\eqref{eq:gauge_2} в последнее уравнение Максвелла:

\begin{equation}
  \label{eq:gauge_5}
  c^2 \vn \times (\vn \times \vec{A}) - \frac{\pt}{\pt t} \left(
  -\vn \phi - \frac{\pt \vec{A}}{\pt t}\right) = \frac{\vec{j}}{\eps_0}.
\end{equation}

Здесь мы учли также, что $\mu_0 \eps_0 =1 /c^2$. Посмотрим на первое
слагаемое в этом уравнении:

\begin{equation}
  \label{eq:gauge_6}
  \vn \times (\vn \times \vec{A}) = \vn (\vn \cdot \vec{A}) - \vn^2 \vec{A}.
\end{equation}

В итоге получим довольно задумчивое соотношение: 

\begin{equation}
  \label{eq:gauge_7}
  -c^2 \vn^2 \vec{A} + c^2 \vn (\vn \cdot \vec{A}) + \frac{\pt}{\pt t}
  \vn \phi + \frac{\pt^2 \vec{A}}{\pt t^2} = \frac{\vec{j}}{\eps_0}.
\end{equation}

И вот сейчас настало время для выбора калибровки. Фактически, мы можем
наложить соотношение на потенциалы без изменения физики --- это будет
эквивалентно фиксации функции $\psi$ в калибровочном преобразовании. 

Выберем калибровку так, чтобы убить второе и третье слагаемое в левой
части уравнения \eqref{eq:gauge_7}:

\begin{equation}
  \label{eq:gauge_fix}
  \vn \cdot \vec{A} = -\frac{1}{c^2} \frac{\pt \phi}{\pt t}.
\end{equation}

При таком выборе уравнение \eqref{eq:gauge_7} становится намного
проще: 

\begin{equation}
  \label{eq:wave_A}
  \vn^2 \vec{A} - \frac{1}{c^2} \frac{\pt^2 \vec{A}}{\pt t^2} = -
  \frac{\vec{j}}{\eps_0 c^2} = -\mu_0 \vec{j}.
\end{equation}

Кроме того, уравнение \eqref{eq:gauge_4} тоже сильно упрощается после
выбора такой калибровки: 

\begin{equation}
  \label{eq:wave_phi}
  \vn^2 \phi - \frac{1}{c^2} \frac{\pt^2 \phi}{\pt t^2} = - \frac{\rho}{\eps_0}.
\end{equation}

Заметим две важные детали. Во-первых, уравнения разделились --- каждое
из них содержит либо $\phi$, либо $\vec{A}$. 

Во-вторых, они имеют абсолютно одинаковую форму (с точностью до
обозначений). Тем самым, решив одно из них, мы легко решим второе. 

Чтобы осознать физический смысл этих уравнений, раскроем оператор
$\vn^2 \phi$ в уравнении \eqref{eq:wave_phi}:

\begin{equation}
  \label{eq:wave_phi_expand}
  \frac{\pt^2 \phi}{\pt x^2} +   \frac{\pt^2 \phi}{\pt y^2} +
  \frac{\pt^2 \phi}{\pt z^2} -  \frac{1}{c^2}  \frac{\pt^2 \phi}{\pt t^2} = - \frac{\rho}{\eps_0}.
\end{equation}

Это уравнение имеет замечательную симметрию по $x,y,z,t$. Множитель
$1/c^2$ перед последним слагаемым появился не зря: он отвечает за
размерности. 

Уравнения \eqref{eq:wave_A} и \eqref{eq:wave_phi} --- по сути,
переписанные уравнения Максвелла. Если мы их решим, то сможем легко
вычислить $\vec{E}$ и $\vec{B}$. Во многих случаях с этими двумя
уравнениями обращаться проще, чем с исходными. 

\subsection{Волновые уравнения в физике. }
\label{sec:wave_equation}

\subsection{Решение волнового уравнения.}
\label{sec:wave_solution}

Рассмотрим для начала простой случай, когда по осям $y,z$ ничего не
меняется. Более того, забудем про все источники: будем считать, что
$\rho=0$ и $\vec{j}=0$. Иными словами, мы имеем уравнения Максвелла в
пустом пространстве (вакууме). 

Так как от координат $y,z$ поля не зависят, то теорема Гаусса даёт

\begin{equation}
  \label{eq:waves_1}
  \frac{\pt E_x}{\pt x}=0.
\end{equation}

Решение этого уравнения --- поле, компонента которого $E_x$ не зависит
от $x$, т.е. является константой. Более того, эта константа равна
нулю. Иными словами, Если движение поля происходит вдоль оси $x$, то
поле располагается поперёк этого направления (его проекция на это
направление равна нулю). 

Рассмотрим теперь оставшееся поперечное поле. Разобьём его на две
компоненты, $y$ и $z$. Разберём сначала случай наличия только одной
компоненты, скажем, по оси $y$. По принципу суперпозиции мы всегда
сможем безболезненно включить другую компоненту. 

Видно, что уравнения Максвелла очень сильно упростятся. Рассмотрим
второе уравнение, которое связывает ротор электрического поля с
изменением магнитного. Компоненты ротора по оси $x$ и $y$ тождественно
равны нулю, а компонента по $z$ равна

\begin{equation}
  \label{eq:waves_2}
  \left(\vn \times \vec{E}  \right)_z = \frac{\pt E_y}{\pt x}.
\end{equation}

Второе уравнение Максвелла тем самым даёт, что

\begin{equation}
  \label{eq:waves_3}
  \frac{\pt B_x}{\pt t} = 0, \quad   \frac{\pt B_y}{\pt t} = 0, \quad
  \frac{\pt B_z}{\pt t} = - \frac{\pt E_y}{\pt x}.
\end{equation}

Получается, что поля $B_x$ и $B_y$ постоянны во времени. Это
постоянство всегда можно свести к нулю (это можно увидеть также из
равенства нулю дивергенции $\vec{B}$). Таким образом, поле $\vec{B}$
также является поперечным: оно направлено перпендикулярно к
направлению движения. 

Кроме того, мы видим, что если электрическое поле имеет ненулевую
$y$--компоненту, то магнитное --- $z$--компоненту. Это означает, что
два этих поля перпендикулярны друг к другу. Заметим, что в разделе
\ref{sec:moving_plate} мы уже наблюдали это явление. Теперь же мы
заметили это для довольно общей ситуации. 

Теперь вспомним про последнее уравнение Максвелла; оно связывает ротор
магнитного поля и динамику электрических полей. В нашем случае оно
сведётся к такому: 

\begin{equation}
  \label{eq:waves_4}
  -c^2 \frac{\pt B_z}{\pt x} = \frac{\pt E_y}{\pt t}.
\end{equation}

Теперь мы можем объединить уравнения \eqref{eq:waves_4} и
\eqref{eq:waves_3} в одно: 

\begin{equation}
  \label{eq:waves_E}
  \frac{\pt^2 E_y}{\pt x^2} - \frac{1}{c^2} \frac{\pt^2 E_y}{\pt t^2}=0.
\end{equation}

Заметим, что это уравнение абсолютно того же типа, что и
\eqref{eq:wave_phi} или \eqref{eq:wave_A} (только, разумеется, с
другими обозначениями). То есть, компоненты электрического поля также
удовлетворяют волновому уравнению. 

Можно показать, что самым общим решением волнового уравнения является
такая комбинация функций: 

\begin{equation}
  \label{eq:waves_general_solution_1}
  E_y = f(x-ct) + g(x+ct),
\end{equation}
где $f,g$ --- произвольные функции. То есть, наше поле может меняться
\textit{как угодно} в зависимости от координаты $x$, и это самое общее
физическое решение. Единственное условие --- зависимость не от $x$ и
$t$ по отдельности, а от комбинаций $x-ct$ и $x+ct$. 

\begin{wrapfigure}{r}{40mm}
  \vspace{-1cm}
  \begin{center}
  \includegraphics[width=4cm]{wave.pdf}  
  \end{center}
  \vspace{-0.5cm}
  \caption{Произвольная волна.}
  \label{fig:wave}
\end{wrapfigure}

Вспоминая теперь про принцип суперпозиции, мы можем включить
зависимость по компоненте $z$. В итоге общее решение для волны,
распространяющейся в направлении $x$, будет выглядеть так: 

\begin{eqnarray}
  \label{eq:waves_general_solution_2}
  \nn
E_x&=&0, \quad E_y = f(x-ct) + g(x+ct), \quad E_z = F(x-ct) + G(x+ct), \\
B_x&=&0, \quad cB_y = f(x-ct) - g(x+ct), \quad cB_z = -F(x-ct) + G(x+ct).
\end{eqnarray}

Это решение характеризует волны, распространяющиеся как в будущее, так
и в прошлое. Мы же ограничимся <<физическим>> решением и везде будем
полагать, что $g=0$.

Ещё раз отметим общие свойства подобного решения: электрическое и
магнитное поля направлены перпендикулярно направлению распространения,
а также перпендикулярно друг к другу. 

\subsection{Вектор Пойнтинга. }
\label{sec:poynting}

На примере подобной электромагнитной волны очень удобно понять
физический смысл вектора Пойнтинга \eqref{eq:def_pointing}, который мы
ввели почти в самом начале, чтобы написать закон сохранения для
электромагнитного поля.

Во-первых, заметим, что в нашей волне $B = E/c$. Во-вторых, так как
магнитное и электрическое поле ортогональны друг другу, то векторное
произведение в определении вектора Пойнтинга превращается в обычное
произведение

\begin{equation}
  \label{eq:poynting_wave_1}
  S = \vec{E} \times \vec{H} = \frac{E B}{\mu_0} = \frac{E^2}{\mu_0 c}
  = \eps_0 c E^2.
\end{equation}

С другой стороны, мы всегда можем вычислить энергию поля по формулам
\eqref{eq:energy_field} и \eqref{eq:def_magnetic_energy}. В нашем
случае, когда магнитное поле равно электрическому с точностью до
скорости света, всё сильно упрощается. Именно, для плотности энергии
электромагнитной волны имеем

\begin{equation}
  \label{eq:poynting_wave_2}
  u = \frac{\eps_0}{2}E^2 + \frac{\mu_0}{2} \frac{E^2}{c^2} = \eps_0 E^2.
\end{equation}

Так как свет распространяется со скоростью $c$, то энергия, проходящая
в секунду через единичную площадку, равна
\begin{equation}
  \label{eq:poynting_wave_3}
  S = \eps_0 c E^2.
\end{equation}

Наблюдаем совпадение формул \eqref{eq:poynting_wave_1} и
\eqref{eq:poynting_wave_3}. Таким образом, вектор Пойнтинга $S$, по
крайней мере в данном случае, действительно имеет смысл плотности
электромагнитной энергии на единицу площади. 

\subsection{Импульс электромагнитного поля. }
\label{sec:field_momentum}

Вектор Пойнтинга позволяет обсудить ещё одно очень любопытное свойство
электромагниного поля, а именно, как электромагнитное поле
воздействует на какое-либо тело. Пусть наша волна падает
перпендикулярно на пластинку. Электрическое поле возбудит в пластинке
ток. По определнию, он равен $\vec{j} = \sigma \vec{E}$. Магнитное
поле, в свою очередь, будет действовать на этот ток по закону Ампера: 
\begin{equation}
  \label{eq:momentum_1}
  \vec{F} = \vec{j} \times \vec{B}.
\end{equation}

По закону Ньютона, эта сила вызывает изменение импульса: $P = F \Delta
t = j B \Delta t$. В этом же слое в единицу времени происходит
поглощение энергии (и выделение в виде тепла) взаимодействия тока с
электрическим полем: $W = j E \Delta t$. Вычисляя отношение
переданного импульса к поглощённой энергии, получаем

\begin{equation}
  \label{eq:momentum_2}
  \frac{P}{W}= \frac{B}{E} = \frac{1}{c}.
\end{equation}

Таким образом, любая энергия $W$ несёт с собой импульс $P=W/c$. В
нашем случае мы знаем, что плотность этой энергии даётся формулой
\eqref{eq:poynting_wave_3}, а значит, переданный импульс на единицу
объёма равен $p=\eps_0 E^2/c$. Несмотря на то, что в знаменателе стоит
очень большая величина (скорость света), померять световое давление
оказывается возможным --- это сделал \textbf{П.Н. Лебедев} в 1900
году, тем самым блестяще подтвердив теорию Максвелла. 

Кроме того, можно получить намёк и на ещё одно интересное
следствие. Представим себе, что свет --- это что-то, движущееся со
скорстью $c$. У этого чего-то будет импульс $P=mc$. Но согласно нашему
утверждению, этому импульсу соответствует какая-то энергия $E$. В
общем, как ни крути, получается, что $E=mc^2$.

Разумеется, это не может являться выводом знаменитой формулы
Эйнштейна, но наши наблюдения имеют непосредственное отношение к этому.

\section{Заряжающийся конденсатор.}
\label{sec:condensator}

Разберём ещё один пример перекачивания энергии переменным
электромагнитным полем. Рассмотрим конденсатор: он состоит из двух
параллельных круговых обкладок радиуса $R$, соединённых с внешним
генератором парой проводов. Если зарядить конденсатор постоянным
током, то на одной из обкладок появится положительный заряд, а на
другой --- отрицательный. Между обкладками будет однородное
электрическое поле. Это --- хорошо известная система. 

\begin{figure}[h]
  \centering
  \subfloat[Контур $\Gamma_1$.]{\label{fig:cond1}\includegraphics[scale=1]{cond1.pdf}}
  \hspace{2cm}
  \subfloat[Контур $\Gamma_2$.]{\label{fig:cond2}\includegraphics[scale=1]{cond2.pdf}}
  \caption{Заряжающийся конденсатор.}
  \label{fig:cond}
\end{figure}

Теперь усложним задачу. Представим себе, что на конденсатор мы подаём
переменное напряжение частоты $\omega$. Запишем поле как $E=E_0 \sin
\omega t$ (мы взяли простейшую периодическую функцию). Разумеется, это
поле создаст за собой магнитное. Вычислим его. Возьмём в качестве
контура $\Gamma_1$ окружность радиуса $r$ между обкладками
конденсатора (см. рис. \ref{fig:cond1}). Тогда четвёртое уравнение Максвелла даст

\begin{equation}
  \label{eq:cond_1}
  c^2 B \cdot 2 \pi r = \frac{\pt}{\pt t} E \cdot \pi r^2.
\end{equation}

Циркуляция магнитного поля по кругу равна изменению потока
электрического поля через этот самый круг. Поле, разумеется, в каждый
момент времени однородно, поэтому поток равен просто произведению
напряжённости на площадь круга. Если взять производную от поля,
получим

\begin{equation}
  \label{eq:cond_2}
  B = \frac{\omega r}{2c^2} E_0 \cos \omega t.
\end{equation}

Итак, магнитное поле тоже колеблется. Но по закону Фарадея это
означает, что электрическое поле не может оставаться однородным!
Изменяющийся магнитный поток должно создать ЭДС. Получается, что наша
картина неверна. Мы можем постараться её поправить. Добавим к
нашему исходному полю $E_1=E_0 \sin \omega t$ поправку
$E_2$. Правильное поле тогда будет записываться как $E=E_1+E_2$. 

Чтобы вычислить $E_2$ надо воспользоваться законом Фарадея. В качестве
контура возьмём $\Gamma_2$ (рис. \ref{fig:cond2}). Напишем 

\begin{equation}
  \label{eq:cond_3}
  \oint_{\Gamma_2} \vec{E} \cdot d \vec{S} = -\frac{\pt \Phi}{\pt t}.
\end{equation}

Заметим, что циркуляция поля $E_1$ по этому контуру даёт ноль; значит,
вклад вносит только поправка $E_2$. Циркуляция этого поля равна
$-E_2(r) h$ (так как по трём сторонам из четырёх она равна
нулю). Закон Фарадея тогда перепишется как

\begin{equation}
  \label{eq:cond_4}
  E_2 (r) = \frac{\pt}{\pt t} \int B(r)\, dr = -\frac{\omega^2
    r^2}{4c^2} E_0 \sin \omega t.
\end{equation}

Итого, наше поле с учётом поправки равно 

\begin{equation}
  \label{eq:cond_5}
  E = E_1 + E_2 = \left( 1 - \frac{1}{4}\frac{\omega^2 r^2}{c^2}
  \right) E_0 \sin \omega t. 
\end{equation}

Электрическое поле в конденсаторе больше неоднородно: оно спадает к
краям параболическим образом. 

Но это ещё не всё. Мы поправили электрическое поле, значит, наш расчёт
для магнитного надо переделывать. Теперь будем искать поправку $B_2$ к
исходному магнитному полю $B_1$. Опять возьмём четвёртое уравнение
Максвелла и контур $\Gamma_1$. По принципу суперпозиции достаточно
подсчитать поток $E_2$ --- это даст циркуляцию добавки $B_2$:

\begin{equation}
  \label{eq:cond_6}
  c^2 B_2 \cdot 2\pi r = \frac{\pt \Phi_2}{\pt t}.
\end{equation}

Теперь поле $E_2$ неоднородно, и взять просто интеграл не получится
--- придётся интегрировать. Ответ для нашей добавки $B_2$ будет таким:

\begin{equation}
  \label{eq:cond_7}
  B_2 (r) = -\frac{\omega^3 r^3}{16 c^3} E_0 \cos \omega t.
\end{equation}

Значит, теперь суммарное магнитное поле равно

\begin{equation}
  \label{eq:cond_8}
  B = B_1 + B_2 = \frac{\omega r }{2c^2} \left(1-\frac{\omega^2 r^2}{8c^2} \right) E_0 \cos \omega t.
\end{equation}

Но дальше можно продолжить эту игру. Мы опять должны поправить
электрическое поле, потом магнитное, и так далее, пока не
надоест. Можно показать (это несложно), что весь ряд для
электрического поля выглядит так: 

\begin{equation}
  \label{eq:cond_9}
  E = E_0 \sin \omega t \left( 1- \frac{1}{(1!)^2} \left( \frac{\omega
      r}{2c} \right)^2 + \frac{1}{(2!)^2} \left( \frac{\omega
      r}{2c} \right)^4 - \frac{1}{(3!)^2} \left( \frac{\omega
      r}{2c} \right)^6 + \ldots \right).
\end{equation}

Очевидно, как продолжить этот ряд. 

То, что написано в скобках, называется \textbf{функцией
  Бесселя}. Функция Бесселя $J_0(x)$ определяется так: 
\begin{equation}
  \label{eq:def_bessel}
  J_0 (x) = 1 - \frac{1}{(1!)^2} \left(\frac{x}{2} \right)^2 +
  \frac{1}{(2!)^2} \left(\frac{x}{2} \right)^4 -  \frac{1}{(3!)^2}
  \left(\frac{x}{2} \right)^6 + \ldots.
\end{equation}

Получается, что наше решение для электрического поля в конденсаторе
равно 

\begin{equation}
  \label{eq:sol_conden}
  E = E_0 \sin \omega t\,  J_0 \left( \frac{\omega r}{c} \right). 
\end{equation}

Если нарисовать график функции $J_0 (x)$, то мы увидим следующее: 

\begin{figure}[h]
  \centering
  \includegraphics[width=10cm]{bessel.pdf}
  \caption{Функция Бесселя.}
  \label{fig:bessel}
\end{figure}

Заметим, что когда $x \approx 2.4$ функция Бесселя обращается в
нуль. То есть, в точках $\omega r \approx 2.4 c$ поле между обкладками
в конденсаторе равно нулю! Более того, как видно из графика,
возможна ситуация, когда поле в центре конденсатора (при $r=0$) и на
краях направлено в противоположные стороны. Это довольно сильно
разнится с нашей первоначальной гипотезой об однородном электрическом
поле. 

\subsection{Перетекание энергии в конденсаторе. }
\label{sec:energy_capacitor}

Теперь посмотрим, как ведёт себя энергия конденсатора, который
находится в этом переменном электромагнитном поле. Если радиус пластин
равен $R$, а расстояние между ними $h$, то полная энергия будет равна

\begin{equation}
  \label{eq:energy_cap}
  U = \frac{\eps_0}{2} E^2 \, \pi R^2 h.
\end{equation}

Очевидно, что раз поле меняется, то меняется и наша энергия. Если
конденсатор заряжается, энергия увеличивается, и должен существовать
поток энергии, направленный откуда-то со стороны внутрь объёма. 

Удивительный факт состоит в том, что этот поток направлен не со
стороны проводов. Действительно, мы видели, что вектор Пойнтинга
(ответственный за направление потока энергии) перпендикулярен как
электрическому, так и магнитному полю. В нашем случае $\vec{E}$
перпендикулярен к пластинам, поэтому вектор Пойнтинга должен быть
параллельно им. 

Кроме того, мы знаем, что вектор $\vec{B}$ направлен по касательной к
окружности с центром на оси конденсатора. Так что вектор Пойнтинга
параллелен пластинам и направлен по радиусу к оси симметрии
конденсатора. Получается, что энергия втекает в конденсатор не по
проводам, а через зазор между пластинами. 


\section{Электротехника.}
\label{sec:electrotechnics}

Будем теперь потихоньку спускаться с небес теории электромагнитного
поля к чему-то земному, например, к электрическим цепям.

\subsection{Генератор переменного тока. }
\label{sec:generator}

Пусть имеется круглая катушка из проволоки, которая может вращаться
вокруг оси, проходящей через один из её диаметров. Оба конца катушки
выведены на внешнюю цепь (например, на амперметр). Катушка помещена в
магнитное поле, перпендикулярное оси вращения. 

Поток через катушку зависит от угла $\theta$, под которым площадка находится по
отношению к магнитному полю: $\Phi = B\cdot S \cdot \cos \theta$. Если
катушка вращается с постоянной угловой скоростью, то $\theta = \omega
t$. По закону Фарадея в катушке возникает ЭДС:

\begin{equation}
  \label{eq:generator_faradey}
  \eps = -\frac{d \Phi}{dt}  = B S \omega \sin \omega t.
\end{equation}

На большом расстоянии от катушки, где $\rot \vec{E} =0$, можно
сказать, что разность потенциалов равна этой ЭДС: $V = B S \omega \sin
\omega t \equiv V_0 \sin \omega t$. Это --- переменная разность
потенциалов. 

Раз есть разность потенциалов, то, очевидно, возникает переменный
ток. Если вся наша цепь имеет сопротивление $R$, то переменный ток
равен

\begin{equation}
  \label{eq:generator_current}
  I = \frac{V}{R} = \frac{V_0}{R} \sin \omega t.
\end{equation}

Покажем теперь, что ЭДС определяет количество энергии, поставляемое
генератором. В единицу времени каждый заряд получает энергию $\vec{F}
\cdot \vec{v}$. Пусть концентрация зарядов на единицу длины провода
равна $n$. Тогда в куске провода длиной $dl$ мощность равна

\begin{equation}
  \label{eq:generator_power_1}
  \vec{F} \ cdot \vec{v} n dl = nv \vec{F} \cdot d\vec{l}.
\end{equation}

Полная мощность, выделяющаяся в цепи, равна интегралу от этой
величины по всей цепи:

\begin{equation}
  \label{eq:generator_power_2}
  P = \oint nv \vec{F} \cdot d\vec{l}.
\end{equation}

Теперь заметим, что $qnv$ --- это ток $I$, а $F/q$ --- ЭДС. Таким
образом, получаем, что $P = \vareps I$.

\subsection{Катушки индуктивности.}
\label{sec:coil}

\subsubsection{Взаимная индукция.}
\label{sec:common_induction}

Теперь рассмотрим случай, когда катушки неподвижны, а меняется
магнитное поле. Сделаем катушку номер 1 в виде длинного соленоида,
обмотанного проволокой. Вокруг этой катушки и изолированно от неё
намотаем катушку номер 2. Если в катушке номер 1 пропустить ток, то он
вызовет магнитное поле. Меняющийся поток этого магнитного поля вызовет
индуцированную ЭДС в катушке номер 2. 

Если в катушке 1 течёт ток $I$, то по формуле
\eqref{eq:mfield_solenoid} возникает магнитное поле. Пусть площадь
сечения катушки 1 равна $S$, а число витков во второй катушке равно
$N_2$. Тогда поток магнитного поля во второй катушке равен $\Phi_2 = B
S N_2$. По закону Фарадея в ней возникает переменная ЭДС, равная

\begin{equation}
  \label{eq:induction_eds}
  \vareps = - N_2 S \frac{dB}{dt} = -\mu_0 \frac{N_1 N_2 S}{l} \frac{dI_1}{dt}.
\end{equation}

Таким образом, ЭДС во второй катушке прямо пропорциональна скорости
изменения тока в первой катушке. Коэффициент пропорциональности
зависит только от геометрии двух катушек. Он называется
\textbf{коэффициентом взаимной индукции}, и обычно обозначается
$L_{12}$.

Попробуем теперь всё провернуть наоборот: пропустить ток через катушку
2 и посмотреть, что будет в катушке 1. Понятно, что ЭДС в катушке 1
будет пропорциональна скорости изменения тока в катушке 2. Коэффициент
пропорциональности можно обозначить как $L_{21}$. Можно показать, что
$L_{12} = L_{21}$ для любых катушек и для любой их конфигурации. 

\subsubsection{Самоиндукция.}
\label{sec:self_induction}

Представим теперь, что ток меняется в обеих катушках
одновременно. Магнитный поток в каждой катушке будет по принципу
суперпозиции представлять сумму потоков. Таким образом, ЭДС в катушке
2, например, можно записать так: 

\begin{equation}
  \label{eq:self_induction_1}
  \vareps_2 = L_{21} \frac{dI_1}{dt} + L_{22} \frac{dI_2}{dt}.
\end{equation}

Аналогично для ЭДС в катушке 1:

\begin{equation}
  \label{eq:self_induction_2}
  \vareps_1 = L_{12} \frac{dI_2}{dt} + L_{11} \frac{dI_1}{dt}.
\end{equation}

Коэффициенты $L_{11},L_{22}$ всегда отрицательны. Обычно их обозначают
просто $-L_1,-L_2$ и называют \textbf{индуктивностями}. Опять же, этот
параметр зависит исключительно от геометрии катушки. Для каждой
отдельной катушки можно записать

\begin{equation}
  \label{eq:self_induction_3}
  \vareps = - L \frac{dI}{dt}. 
\end{equation}

Знак минус указывает на то, что ЭДС противодействует изменению тока. 

Поскольку имеется подобное противодействие, ток в катушке обладает
некоторой инерцией. Если мы захоим изменить ток в катушке, надо
преодолеть эту инерцию, присоединив катушку ко внешнему генератору. В
такой цепи ток будет связан с напряжением соотношением

\begin{equation}
  \label{eq:self_induction_4}
  V = L \frac{dI}{dt}.
\end{equation}

Это уравнение имеет вид закона Ньютона для частицы в одном
измерении. Имеет место простая аналогия: сила $F$ соответствует
разности потенциалов $V$, скорость $v$ --- силе тока $I$, масса $m$
--- индуктивности $L$ (именно поэтому катушка обладает <<инерцией>>
--- совсем как массивное тело). 

Продолжая эту аналогию, можно выяснить, что энергия такой цепи будет
равна $LI^2/2$. Фактически, мы вывели энергию цепи с катушкой исходя
только лишь из электромеханических аналогий. Кроме того, можно
показать, что для системы двух связанных катушек (которые мы
рассматривали в предыдущем разделе), энергия будет равна

\begin{equation}
  \label{eq:self_induction_energy}
  E = \frac12 L_1 I_1^2 + \frac12 L_2 I_2^2 - L_{12} I_1 I_2.
\end{equation}

Действительно, пускай сначала токи в обеих катушках равны
нулю. Включим сначала ток $I_1$, оставляя $I_2=0$. Совершённая работа
как раз равна $L_1 I_1^2/2$. Теперь, включая $I_2$, мы совершаем
работу $L_2 I_2^2/2$ не только против ЭДС в катушке 2, но и против ЭДС
в катушке 1. Эта последняя работа равна произведению $I_1$ на ЭДС в
катушке 1. ЭДС в катушке 1 равна интегралу от $L_{12} dI_2/dt$. Таким
образом, получаем формулу \eqref{eq:self_induction_energy}. 


Наконец, вычислим индуктивность одиночной катушки. Проще всего это
сделать из энергетических соображений. Действительно, пусть в катушке
магнитное поле равно $B$. Тогда энергия, запасённая в катушке радиусом
$r$ и длиной $l$ равна

\begin{equation}
  \label{eq:coil_energy_1}
  U = \frac{1}{2 \mu_0} B^2 \Rightarrow E = \frac{1}{2\mu_0} (\mu_0 n
  I)^2 \pi r^2 l = \frac{\mu_0}{2} \pi r^2 l n^2 I^2.
\end{equation}

С другой стороны, эта энергия равна $LI^2/2$. Отсюда получаем, что
индуктивность одиночной катушки равна

\begin{equation}
  \label{eq:inductivity}
  L = \mu_0 \pi r^2 l n^2.
\end{equation}

\subsection{Законы Кирхгофа.}
\label{sec:kirchoff}

Рассмотрим цепь, состоящую из произвольного набора элементов:
конденсаторов, катушек, резисторов и генераторов. Примём одно важное
допущение: меняющееся магнитное поле ограничено только малой областью
в окрестности приборов, в остальном пространстве его нет.

Рассмотрим кривую $\Gamma$, обходящую цепь вокруг приборов. Контурный
интеграл от $\vec{E}$ вдоль этой кривой состоит из нескольких
частей. Каждая часть --- это интеграл от одного зажима прибора до
другого. Мы называем этот интеграл \textit{падением напряжения} на
приборе. Тогда весь интеграл равен просто сумме падений напряжений:

\begin{equation}
  \label{eq:kirchoff_1}
  \oint \vec{E} \cdot d \vec{l} = \sum_n V_n.
\end{equation}

Так как меняющихся магнитных полей за пределами приборов нет, то этот
интеграл равен нулю во второму уравнению Максвелла. Следовательно, 

\begin{equation}
  \label{eq:kirchoff_2}
  \sum_n V_n = 0. 
\end{equation}

Предположим теперь, что зарядом, накапливающимся на приборах, можно
пренебречь. Тогда сохранение заряда требует, чтобы любой заряд,
вышедший из одного элемента цепи, входил в другой элемент. Или, что то
же самое, алгебраическая сумма токов, входящих в любую точку
соединения, должна быть равна нулю:

\begin{equation}
  \label{eq:kirchoff_3}
  \sum_n I_n =0.
\end{equation}

Уравнения \eqref{eq:kirchoff_2} и \eqref{eq:kirchoff_3} называются
\textbf{правилами Кирхгофа}. С их помощью можно найти силы токов и
напряжения в любой цепи. 

Заметим, что эти уравнения мы получили из уравнений Максвелла и из
естественного требования сохранения заряда. То есть, правила Кирхгофа
довольно фундаментальны. Они справедливы для любых цепей, при
переменном и постоянном токе. 

\begin{figure}[h]
  \centering
  \includegraphics[scale=1.5]{bridge}
  \vspace{-1cm}
  \caption{Мостовая схема.}
  \label{fig:bridge}
\end{figure}

В качестве примера рассмотрим схему моста c сопротивлениями $R_1, R_2,
R_3, R_4, R_5$. Определим его результирующее сопротивление. Иными
словами, какой будет ток на зажимах моста, если мы подадим на него
разность потенциалов $U$?

Рассмотрим узел $A$. В нём должен выполняться второй закон
Кирхгофа:

\begin{equation}
  \label{eq:apply_kirchoff_0}
  I = I_1 + I_2
\end{equation}

В петле $ADB$ должен выполняться первый закон Кирхгофа:

\begin{equation}
  \label{eq:apply_kirchoff_1}
  R_1 I_1 + R_3 I_3 -R_2 I_2 = 0.
\end{equation}

В петле $DCB$ также:

\begin{equation}
  \label{eq:apply_kirchoff_2}
  R_4 (I_1 - I_3) - R_5 (I_2 + I_3) -R_3 I_3 =0.
\end{equation}

Здесь мы учли второй закон Кирхгофа в узлах $B,D$. Падение потенциалов на
мосту равно 

\begin{equation}
  \label{eq:apply_kirchoff_3}
  U_{AC} = U_{AD} + U_{DC} = R_1 I_1 + R_4 (I_1 -I_3).
\end{equation}

Из уравнений \eqref{eq:apply_kirchoff_0}, \eqref{eq:apply_kirchoff_1}
и \eqref{eq:apply_kirchoff_2} можно выразить $I_1, I_2$:

\begin{eqnarray}
  \label{eq:apply_kirchoff_currents}
  \nn
  I_1 = -\frac{I_3 (R_3 R_5 + R_2 (R_3 + R_4 + R_5))}{R_2 R_4 -R_1
    R_5},\\
  I_2 = -\frac{I_3 (R_3 R_4 + R_1 (R_3 + R_4 + R_5))}{R_2 R_4 -R_1
    R_5}.
\end{eqnarray}

Общий ток, проходящий через мост равен $I=I_1 + I_2$ и также
выражается через $I_3$. С разностью потенциалов такая же штука
(см. \eqref{eq:apply_kirchoff_3}). Разделив одно на другое мы
избавимся от $I_3$ и в сухом остатке получим комбинацию сопротивлений:

\begin{equation}
  \label{eq:bridge_answer}
  R = \frac{R_1 R_2 R_3 + R_1 R_2 R_4 + R_2 R_3 R_4 + R_1 R_2 R_5 +
    R_1 R_3 R_5 + R_1 R_4 R_5 + R_2 R_4 R_5 + R_3 R_4 R_5}{R_1 R_3 + R_2 R_3 + R_1 R_4 + R_2 R_4 + R_3 R_4 + R_1
    R_5 + R_2 R_5 + R_3 R_5}.
\end{equation}

Мы получили ответ для \textit{произвольного} (а не только
сбалансированного) моста, решив всего лишь небольшую систему
уравнений. Правила Кирхгофа нам здорово помогли. 

\section{Оптика.}
\label{sec:optics}

\subsection{Интерференция.}
\label{sec:interference}

В качестве стартовой рассмотрим такую задачу: пусть имеется два точечных источника, излучающие гармоническую волну с одинаковой частотой $\omega$. Каково будет распределение поля, создаваемое этими источниками?
Волны записываются в виде комплексных экспонент:
\begin{equation}
W_1=A_1 e^{i(\omega t + \phi_1)}, \quad W_2 = A_2 e^{i(\omega t + \phi_2)}.
\end{equation}
Сложим эти две волны:
\begin{equation}
R = A_1 e^{i(\omega t + \phi_1)} + A_2 e^{i(\omega t + \phi_2)} = (A_1 e^{i \phi_1} + A_2 e^{i \phi_2} ) e^{i \omega t}.
\end{equation}
В принципе, это готовый ответ. Найдем амплитуду $R$, то есть $A_R$. 
\begin{equation}
\begin{split}
A_R^2 = R R^{*} = (A_1 e^{i \phi_1} + A_2 e^{i \phi_2} ) & (A_1 e^{-i \phi_1} + A_2 e^{-i \phi_2} ) =\\
=A_1^2 + A_2^2 + A_1 A_2 \left[ e^{i (\phi_1 - \phi_2)} + e^{i (\phi_2 - \phi_1)} \right] &= A_1^2 + A_2^2 + 2 A_1 A_2 \cos (\phi_2 - \phi_1).
\end{split}
\end{equation}

Назовем \textit{интенсивностью} $I$ квадрат амплитуды электромагнитного поля. Таким образом, мы можем записать наш результат в виде:
\begin{equation} \label{eq:sum_waves}
I = I_1 + I_2 + 2\sqrt{I_1 I_2} \cos (\phi_2 - \phi_1).
\end{equation}

\begin{figure}[h]
  \centering
  \begin{tikzpicture}[>=latex]
    \coordinate (a) at (1,0);
    \coordinate (b) at (1,4);
    \coordinate (c) at (3,0.5);
    \draw[thick] (a) -- (b);
    \filldraw[black] (a) circle (0.1);
    \filldraw[black] (b) circle (0.1);
    \draw (1,2) -- ++(6,0);
    \draw[very thick,->] (1,4) -- ($(1,4)!2.5!(3,4.5)$) node[above,midway,sloped] {К точке P};
    \draw[very thick,->,yshift=-2cm] (1,4) -- ($(1,4)!2.5!(3,4.5)$);
    \draw[very thick,->,yshift=-4cm] (1,4) -- ($(1,4)!2.5!(3,4.5)$);
    \draw (2.5,2) arc (0:atan(1/4):1.5cm) node[below=0.15cm,right]
    {$\theta$};
    \draw[red] (b) -- ($(a)!(b)!(c)$);
    \draw[blue,thick,->] (2.5,-0.5) node[black,right] {$d
      \sin \theta$} to [out=180,in=-60] (1.5,0);
    \draw (a) -- ++(-1.2,0);
    \draw (b) -- ++(-1.2,0);
    \draw[very thick,<->] (0,0) -- ++(0,4) node[midway,left] {$d$};
    \draw[blue,thick,->] (-0.5,5) node[black,above] {$A\sin (\omega t + \alpha)$} to [out=-90,in=135] ($(b)+(-0.2,0.2)$);
    \draw[blue,thick,->] (-0.5,-1) node[black,below] {$A\sin \omega t$} to [out=90,in=225] ($(a)+(-0.2,-0.2)$);
  \end{tikzpicture}
  \caption{Разность хода лучей.}
  \label{fig:two_paths}
\end{figure}

% \pic{\psfrag{A2}{$A e^{i (\omega t + \alpha)}$} \psfrag{A1}{$Ae^{i \omega t}$} \psfrag{theta}{$\theta$} \psfrag{dsint}{$d\sin \theta$} \psfrag{d}{$d$} \psfrag{point}{к точке $P$}}{1}{fig/7}{r}{-5mm}{3mm}
Тут есть сумма $I_1 + I_2$ и дополнительный член, отвечающий за
интерференцию. С ним-то и связано все самое интересное. Применим эту
формулу в следующей ситуации: источники расположены на расстоянии $d$
друг от друга, разность фаз постоянна и равна $\alpha$, амплитуды
также одинаковы. Разность расстояний от точки $P$ до осцилляторов
равна $d\sin \theta$ (см. рис. \ref{fig:two_paths}). Тогда разность фаз
из-за разности хода равна

\begin{equation}
\Delta \phi = \frac{2\pi d \sin \theta}{\lambda}
\end{equation}
Полная разность фаз в точке $P$ равна:

\begin{equation}
\phi_2 - \phi_1 = \alpha + \frac{2\pi d \sin \theta}{\lambda}
\end{equation}

Это выражение (а также равенство $A_1 = A_2 \equiv A_0$) можно подставить в (\ref{eq:sum_waves}) --- получится финальная формула. Зависимость интенсивности от угла $\theta$ дается формулой:

\begin{equation}
I = 4I_0 \cos^2 \left( \frac{\phi_2 - \phi_1}{2} \right) = 4I_0 \cos^2 \left( \frac{\alpha}{2} + \frac{\pi d \sin \theta}{\lambda} \right) 
\end{equation}
В практически важных случаях экран находится достаточно далеко, так что все углы наблюдения малы ($\theta \ll 1$). В этом случае можно заменить $\sin \theta$ на $\theta$. Если $l$ --- расстояние до экрана, то $\theta \approx \dfrac{x}{l}$, где $x$ --- координата вдоль экрана.

Графически (в частном случае $\alpha = 0$, $d = 20 \lambda $) это выглядит так: 

\begin{figure}[h]
  \centering
  \includegraphics[scale=1]{interference_plot.pdf}
\end{figure}

Обратим внимание, что этот график уже интересен: на далеком экране мы
увидим чередующиеся светлые (интенсивностью $4I_0$) и темные (минимумы
на графике) полосы.

\subsection{Опыт Юнга с двумя щелями.}

Пока мы занимались исследованием абстрактной задачи~---~сложением двух
монохроматических волн. Где подобная ситуация может встретиться в
физике? Впервые этим вопросом задался Юнг, когда провел свой
знаменитый опыт с двумя щелями.

\begin{figure}[h]
  \centering
  \begin{tikzpicture}
    \draw(0,3) -- ++(8,0);
    \draw[thick] (1,6) -- ++(0,-2.75);
    \draw[thick] (1,2.75) -- ++(0,-2.75);
    \foreach \y in {1,1.5,2,2.5,3,3.5,4,4.5,5} {
      \draw[thick,->] (0,\y) -- ++(0.75,0);    
    }
    \draw[thick] (3,6) -- (3,4.75) ++ (0,-0.5) -- (3,1.75) ++(0,-0.5) --
    (3,0);
    \draw[thick,->] (8,6) -- (8,0) node[right] {$x$};
    \draw[very thick] (1,3) node[above=15,right] {$S$} -- (3,4.5) node[above left] {1} -- (8,5.5) node[right] {$P$};
    \draw[very thick] (1,3) -- (3,1.5) node[below left] {2} -- (8,5.5);
    \draw[blue,dashed,thick] (3,3) -- (8,5.5);
    \draw[red,thick] (3,4.5) -- ($(3,1.5)!(3,4.5)!(8,5.5)$);

    \draw[blue,thick,<->] (3,0.5) -- (8,0.5) node[midway,below] {$D$};
    \draw (5,5.3) node {$r_1$};
    \draw (6,3.4) node {$r_2$};
  \end{tikzpicture}
  \caption{Эксперимент Юнга.}
  \label{fig:yung}
\end{figure}

Суть опыта состоит в следующем. На щель $S$ падает плоская
монохроматическая волна (яркий пучок солнечных лучей). Полученный
когерентный ($\alpha = 0$) пучок падает на экран с двумя щелями 1 и 2,
которые расположены достаточно близко друг к другу (но расстояние
между щелями должно быть велико по сравнению с шириной каждой
щели). Ввиду общности происхождения, пучки из щелей 1 и 2 когерентны,
поэтому мы в точности попадаем в ситуацию, описанную в предыдущем
пункте.

Оценим расстояние между светлыми полосами для реального
эксперимента. Пусть расстояние $d$ между щелями составляет 1 мм, а $D$
= 1 м. Пусть мы светим красным светом ($\lambda$ = 600
нм). Зависимость $I(x)$ дается формулой
\begin{equation}
I(x) = 4 I_0 \cos^2 \left( \frac{\pi d x}{D \lambda} \right)
\end{equation}

Максимумы интесивности появляются тогда, когда аргумент косинуса равен
целому кратному числа $\pi$, то есть $x=\cfrac{n D \lambda}{d}, \mbox{
} n \in \mathbb{Z}$. Тогда расстояние между соседними максимумами (или
ширина полосы) равно $\Delta x = \cfrac{D \lambda}{d}$. В наших
условиях оно равно 0.6 мм. Это расстояние вполне можно померять, и,
тем самым, это дает возможность (довольно неточно) померять длину
волны света. Замечание в сторону: если вместо Солнца в качестве
источника использовать лазер, то щель $S$ можно убрать. Щели можно
получить так: достаточно стеклянные пластинки покрыть слоем серебра
(или фольги). Далее на серебряном слое бритвой проводятся две близко
расположенные параллельные линии, которые и используются в качестве
щелей.

\subsection{Два простых эксперимента II. Плоскопараллельная пластинка.}
Еще один интересный эффект связан с прохождением света через тонкую стеклянную пластинку. 

\begin{figure}[h]
  \centering
  \begin{tikzpicture}[decoration={%
      markings,%
      mark=at position 0.65 with {\arrow[black]{latex};}}]
    \draw[fill=gray!10,draw=gray!10] (0,4) rectangle ++(8,-2);
    \draw[thick] (0,4) -- ++(8,0);
    \draw[thick] (0,2) -- ++(8,0);
    \draw[dashed,blue] (2,5) -- ++(0,-3);
    \draw[red,thick] (5,2) -- ($(3,2)!(5,2)!(5,0)$) node[black,below left] {$D$};
    \draw[postaction=decorate,very thick] (0,6) -- (2,4);
    \draw[postaction=decorate,very thick] (2,4) -- (3,2) node[below=10,left=-3] {$A$};
    \draw[postaction=decorate,very thick] (3,2) -- (4,4) node[above] {$B$};
    \draw[postaction=decorate,very thick] (4,4) -- (5,2) node[above
    right] {$C$};
    \draw[postaction=decorate,very thick] (5,2) -- (6,1);
    \draw[postaction=decorate,very thick] (6,1) -- (6.25,-0.25);
    \draw[postaction=decorate,very thick] (3,2) -- (5,0);
    \draw[postaction=decorate,very thick] (5,0) -- (6.25,-0.25)
    node[right] {$P$};
    \draw[blue,thick,<->] (6.5,1.5) -- ++(-135:2.9cm) node[black,below]
    {\footnotesize{линза}};
    \draw (0.5,3) node {$n$};
    \draw[blue] (2,3.25) arc (-90:-90+atan(0.5):0.75);
    \draw[blue] (2.2,3.1) node {$\alpha$};
  \end{tikzpicture}
  \caption{Плоскопараллельная пластинка.}
  \label{fig:plate}
\end{figure}

Рассмотрим луч, падающий на пластинку толщиной $h$ с показателем
преломления $n$. В точку $P$ (фокальная точка зрительной трубы)
приходят много лучей: однократно преломленный, двукратно преломленный
и т.д. Очевидно, что если разность хода первого луча равна $\Delta x$,
то у второго луча --- $2 \Delta x$ и т.д. Тем самым, надо только
подсчитать $\Delta x$.

Из геометрических соображений очевидно, что 

\begin{equation}
\Delta x = n (AB + BC) - AD.
\end{equation}

$AB = BC = 2h / \cos \theta_2$, $AD = 2 h \tg \theta_2 \sin \theta_1$. Отсюда 

\begin{equation}
\Delta x = 2 n h \cos \theta_2.
\end{equation}

Разность фаз тогда равна, очевидно:

\begin{equation}
\delta = k \Delta x = \frac{4 \pi n h \cos \theta_2}{\lambda_0}.
\end{equation}

Каждый раз при отражении от стенок пластинки интенсивность меняется в
$r$ раз ($0<r<1$ --- отражательный коэффициент). Заметим, что каждый
выходящий из пластинки луч испытывает отражение четное число раз. Так
что интенсивности образуют геометрическую прогрессию: $E, \mbox{ }
Er^2, \mbox{ } Er^4, \ldots$. Таким образом, надо просуммировать
следующую прогрессию:
\begin{equation}
A = E \sum_{i=0}^{\infty} r^{2i} e^{i \delta} = E \frac{1}{1 - r^2 e^{i \delta}}.
\end{equation}
(здесь мы воспользовались суммой геометрической прогрессии, так как
$r^2 <1$). Опять, мы должны посчитать амплитуду этого выражения, то
есть, $AA^{*}$:
\begin{equation}
I =E^2 \frac{1}{1 - r^2 e^{i \delta}} \frac{1}{1 - r^2 e^{-i \delta}} = \frac{1}{1 - 2r^2 \cos \delta + r^4}.
\end{equation}

В знаменателе добавим и вычтем $2r^2$. Упрощая, получим:
\begin{equation}
I = \frac{E^2}{(1-r^2)^2 + 4 r^2 \sin^2 \cfrac{\delta}{2}}.
\end{equation}

Для наглядности нарисуем график этой замысловатой функции:

\begin{center}
  \begin{tikzpicture}
    \begin{axis}[ymin=0,ymax=12,grid=major,height=8cm,width=8cm,xlabel={$\delta$},ylabel={$I(\delta)$},axis
      equal]
      \addplot[samples=100,domain=0:12]
      gnuplot[id=sin]{6/(0.75*0.75+sin(x/2)*sin(x/2))};
    \end{axis}
  \end{tikzpicture}
\end{center}

Условия максимума интенсивности можно записать в виде $2nh \cos
\theta_2 = m \lambda_0$, $m \in \mathbb{Z}$. Видно, что линиям равной
интенсивности соответствует одно и то же значение $\theta_1$, так что
интерференционные полосы в фокальной плоскости линзы --- просто
концентрические кольца с центром на оси линзы.



\end{document}
